
\begin{abstract}

% Motivation
        Fault-tolerance poses a major challenge for future large-scale systems.
        Active research in the field typically focuses on mitigating the
        effects of uncorrectable errors, those fatal errors that typically
        require an application to restart from a known good state.
% Methods/Procedures
        However, the impacts of the most common error of large-scale systems,
        correctable errors, is typically overlooked.  Moreover, increased
        memory volumes and expected technology changes on future extreme-scale
        system may make these errors even more likely and more of a concern.
        In this work, we use a simulation-based approach to show how local
        correctable errors can significantly affect the performance of key
        extreme-scale workloads.
% Findings
        Our study shows that local delays due to correctable errors can
        propagate through MPI message synchronization to other processes,
        causing a cascading series of delays.  We also find that though much of
        the focus on correctable errors is focused on reducing failure rates,
        reducing the rate of each individual error may be more impacting on
        overheads at scale. Finally,  this study outlines the errors
        frequencies, durations, and scales in which performance is
        significantly impacted for a number of key extreme-scale workloads.
% Conclusion/Impacts
        This work provides critical analysis and insight into the overheads of
        common correctable errors and provides practical advice to users and
        systems administrators in an effort to fine-tune performance to
        application and system characteristics.  \end{abstract}

