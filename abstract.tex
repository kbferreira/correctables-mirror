
\begin{abstract}

% Motivation
        Fault tolerance poses a major challenge for future large-scale systems.
        Current research on fault tolerance has been principally focused on mitigating the 
        impact of uncorrectable errors: errors that corrupt the state of the machine and
        prevent continued execution.  Recovery from an uncorrectable error typically 
        requires the affected application to restart from a known good state.
% Methods/Procedures
        However, correctable errors occur much more frequently than uncorrectable errors.
        Projections about next-generation large-scale systems --- e.g., larger total memory
        volume, technology changes --- suggest that correctable errors will be more even more 
        common on future systems.  Although an application can safely continue to execute when 
        correctable errors occur, recovery from a correctable error requires the error to be 
        corrected and, in most cases, information about its occurrence to be logged.  The
        potential performance impact of these recovery activities has not been extensively  
        studied.  
% Findings

        In this paper, we use simulation to examine the relationship 
        between recovery from correctable errors and application performance for several important
        extreme-scale workloads.  Our paper contains what is, to the best of our knowledge, the first 
        detailed analysis of the impact of recovering from correctable errors on application 
        performance.  \sll{This \emph{feels} like it's overemphasizing noise.  I \emph{think} that 
        most of the results don't show much propagation/aggregation of noise}
        Our study shows that local delays due to correctable errors can
        propagate through MPI message synchronization to other processes,
        causing a cascading series of delays.  We also find that although 
        the focus on correctable errors is focused on reducing failure rates,
        reducing the rate of each individual error may be more impacting on
        overheads at scale. Finally,  this study outlines the errors
        frequencies, durations, and scales in which performance is
        significantly impacted for a number of key extreme-scale workloads.
% Conclusion/Impacts
        \sll{Advice to users?  What advice is that?}
        This paper provides critical analysis and insight into the overheads of
        correctable errors and provides practical advice to users,
        systems administrators, and hardware designers in an effort to fine-tune performance to
        application and system characteristics.  \end{abstract}

