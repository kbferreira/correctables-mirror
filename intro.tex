
\section{Introduction}
\label{sec:intro}

Maintaining the performance of high-performance computing~(HPC) applications as
failures become more and more frequent is a major challenge that needs to be
addressed for next-generation extreme-scale systems.  Many recent studies have
demonstrated that hardware failures are expected to become ever more
common~\cite{Bergman08exascalecomputing}.  Increasing the scale of HPC systems
requires the aggregation of larger number of individual components.  More
components means more frequent failures.  Current systems use powerful
error-correcting codes~(ECC), e.g., chipkill-correct, to protect against DRAM
errors.  However, chipkill-correct (and other similar techniques) require the
activation of a large number of memory devices (four times more than
less-protective techniques like single error correct double error
detect~(SECDED))~\cite{Jian13}.  Activating more memory devices requires more
power for each memory access.  However, because of tightening power budgets on
next-generation systems~\cite{Bergman08exascalecomputing}, it is not yet clear
that chipkill-correct will continue to be viable.  Reduced device-feature sizes
also have the potential to result in more frequent failures.  Understanding the
implications of these trends requires that we have detailed knowledge of how
failures affect current leadership-class systems.

Recent works typically focus on uncorrectable errors, those errors that cause
applications to restart.  However, the impacts of the most common error of
large-scale systems, correctable errors, is typically overlooked. An analysis of
failures on recent leadership-class systems show that the correctale failure
rate is a factor \hl{ADD} more frequently than uncorrectale errors.  Correctable
errors are typically handled at a hardware level, and not reflected to the
application.  While the application can continue to make progress in the
presence of these correctale errors, the mechcanism to correct and log these
errors have the potential to impact application performance.
