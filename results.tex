
\section{Results}
\label{sec:results}

In this section we outline the costs and impacts of DRAM correctable errors
using the test infrastructure structure outlined in the previous sections.  First,
we examine the correction, logging and decoding costs of these correctable
errors.  This characterization defines the cost of each DRAM error under a number
of different configuration.  Then, using this correctable error cost
characterization and the fault rate measured in a number of published studies as
a baseline, we demonstrate the performance impacts of correctable errors on
current and future systems.

\subsection{DRAM Correctable Costs}

In this section we outline the cost of DRAM correctable on a modern HPC system.
We are concerned with three costs: the hardware error correction costs (the cost
of the ECC code to correct), logging/decoding in the OS, and logging/decoding in
firmware.

To carry out this testing, we use the \blake system located at \detail{Sandia
National Laboratories}.  \blake is 48 node Intel Skylake system connected by
\kbf{ADD network detail}.  Each node consists of 4, 24 core, 2.1GHz Intel
Skylake processors (a total of 96 cores/node) and 175GB of DDR4 DRAM per node.
This system is running Red Hat Enterprise Linux Server release 7.4 and a Linux
3.10.693 version kernel.

To inject correctable DRAM errors we use the Error INJection (EINJ) facility of
the ACPI Platform Error Interface (APEI)~\cite{ACPISpec} described previously.
We also use the \selfish operating system noise (or \emph{jitter}) measurement
microbenchmark~\footnote{See appendix \Cref{sec:appendix} for further details on
these utilities}.

\begin{figure*}
\centering{
        \subfloat[Native OS Signature for \blake]{
                \includegraphics[ width=0.45\textwidth ]
                        {blake-native-selfish-cpu0_dtr}
                \label{blake:native}
        }
        \subfloat[``Dry Run'' Injection OS Signature ]{
                \includegraphics[ width=0.45\textwidth ]
                        {blake-10sec-dry_run-selfish-cpu0_dtr}
                \label{blake:dry-run}
        }
}
\caption{
        Native and ``dry run'' OS noise signature for \blake.  The ''dry run''
        option configures the EINJ interface at the requested frequency (in this
        case every ten seconds) but does not trigger the error.  This attempts
        to measure the cost of the error injection utility.  As can be seen from
        the figure, the injection utility impacts no additional OS noise
}
\label{fig:baselines}
\end{figure*}

As we are interested in measuring the noise signature of DRAM correctables, we
first must measure the native noise signature of the system, as well as ensuring
the error injection utility does not impart additional OS noise costs that would
not be part of an actual correctable DRAM error.  \Cref{fig:baselines} shows the
native OS noise signature on the \blake system and a ``dry-run'' execution of
our error injection utility.  The dry-run option of the injection utility
configures the injection interface to triggers errors at the requested interval,
in this case we arbitrarily chose every 10 seconds, but does not trigger the
error.  An OS noise event detected by \selfish is denoted by a spike in the
figure, the X-Axis being the time the noise event occurred and the duration of
the noise event being the amplitude on the Y-Axis.  From this figure we can see
that the native noise and dry-run signature are nearly identical.  Therefore the
injection framework does not impart any additional OS noise over the systems
native signature.  

\begin{figure*}
\centering{
        \subfloat[Hardware Cost (All Logging Off)]{
                \includegraphics[ width=0.3\textwidth ]
                        {blake-all_loggin_off-selfish-cpu0_dtr}
                \label{blake:off}
        }
        \subfloat[Software Cost (OS decoding)]{
                \includegraphics[ width=0.3\textwidth ]
                        {blake-10sec-edac-ignore_ce-log-correctable-selfish-cpu0_dtr}
                \label{blake:OS_log}
        }
        \subfloat[Firmware Cost (Firmware decoding, threshold set to 10)]{
                \includegraphics[ width=0.3\textwidth ]
                        {blake-10sec-extlog-ignore_ce-selfish-cpu0_dtr}
                \label{blake:FW_log}
        }
}
\caption{
        \kbf{ADD CAPTION}
}
\label{fig:DRAM_cost}
\end{figure*}

\kbf{Add details for \Cref{fig:DRAM_cost}}

\subsection{Exploring Correctable Costs}  

\kbf{Add plotshere ... and an explanation}

\begin{figure*}
\centering{
    \subfloat[Correctables at current rate, 16KNodes]{
        \includegraphics[width=0.3\textwidth]{1.78571e-06_16384-apps-delta.pdf}
        \label{fig:apps:current}
    }
    \subfloat[Correctables at {\bf{10}}X the current rate, 16KNodes]{
        \includegraphics[width=0.3\textwidth]{1.78571e-05_16384-apps-delta.pdf}
        \label{fig:apps:10xcurrent}
    }
    \subfloat[Correctables at {\bf{100}}X the current rate, 16KNodes]{
        \includegraphics[width=0.3\textwidth]{0.000178571_16384-apps-delta.pdf}
        \label{fig:apps:100Xcurrent}
    } \\
    \subfloat[Correctables at {\bf{1000}}X the current rate, 16KNodes]{
        \includegraphics[width=0.3\textwidth]{0.00178571_16384-apps-delta.pdf}
        \label{fig:apps:1000Xcurrent}
    }
    \subfloat[Correctables at {\bf{10,000}}X the current rate, 16KNodes]{
        \includegraphics[width=0.3\textwidth]{0.0178571_16384-apps-delta.pdf}
        \label{fig:apps:10000Xcurrent}
    }
    \subfloat[Correctables at {\bf{100,000}}X the current rate, 16KNodes]{
        \includegraphics[width=0.3\textwidth]{0.178571_16384-apps-delta.pdf}
        \label{fig:apps:100000Xcurrent}
    }
}
\caption
{
        \kbf{Add details}
}
\label{fig:apps-delta:16K}
\end{figure*}

\begin{figure*}
\centering{
    \subfloat[Correctables at current rate, 16KNodes]{
        \includegraphics[width=0.3\textwidth]{1.78571e-06_16384-micros-delta.pdf}
        \label{fig:micros:current}
    }
    \subfloat[Correctables at {\bf{10}}X the current rate, 16KNodes]{
        \includegraphics[width=0.3\textwidth]{1.78571e-05_16384-micros-delta.pdf}
        \label{fig:micros:10xcurrent}
    }
    \subfloat[Correctables at {\bf{100}}X the current rate, 16KNodes]{
        \includegraphics[width=0.3\textwidth]{0.000178571_16384-micros-delta.pdf}
        \label{fig:micros:100Xcurrent}
    } \\
    \subfloat[Correctables at {\bf{1000}}X the current rate, 16KNodes]{
        \includegraphics[width=0.3\textwidth]{0.00178571_16384-micros-delta.pdf}
        \label{fig:micros:1000Xcurrent}
    }
    \subfloat[Correctables at {\bf{10,000}}X the current rate, 16KNodes]{
        \includegraphics[width=0.3\textwidth]{0.0178571_16384-micros-delta.pdf}
        \label{fig:micros:10000Xcurrent}
    }
    \subfloat[Correctables at {\bf{100,000}}X the current rate, 16KNodes]{
        \includegraphics[width=0.3\textwidth]{0.178571_16384-micros-delta.pdf}
        \label{fig:micros:100000Xcurrent}
    }
}
\caption
{
        \kbf{Add details}
}
\label{fig:micros-delta:16K}
\end{figure*}

\subsection{Impact of Scale}

\kbf{Trim down all these figures ... and discuss results.  Also check which
applications are here.  Prolly remove GTC and add SPARC}

\begin{figure*}
\centering{
    \subfloat[Hardware Correction Impacts]{
        \includegraphics[width=0.3\textwidth]{allHz_1.5e-7secs}
        \label{fig:current:hardware}
    }
    \subfloat[Software-based Logging Impacts]{
        \includegraphics[width=0.3\textwidth]{1.7857e-06Hz_0.00077secs}
        \label{fig:current:cmca}
    }
    \subfloat[Firmware-based Logging Impacts]{
        \includegraphics[width=0.3\textwidth]{1.7857e-06Hz_0.13secs}
        \label{fig:current:firmware}
    }}
   \caption{\textbf{Performance impacts of correctable errors using the current
                    correctable error rate from \cielo.  Three scenarios are shown:
                    hardware only correction with no logging ($150ns$ per event),
                    Software-based logging using the Corrected Machine Check
                    Architecture (CMCA) ($775\mu sec$ per event), and the
                    Firmware-based logging
                    using the Enhanced Machine Check Architecture (EMCA)
                    ($133msecs$ per event) }.
   }
\label{fig:current}
\end{figure*}

\begin{figure*}
\centering{
    \subfloat[Hardware Correction Impacts]{
        \includegraphics[width=0.3\textwidth]{allHz_1.5e-7secs}
        \label{fig:10X:hardware}
    }
    \subfloat[Software-based Logging Impacts]{
        \includegraphics[width=0.3\textwidth]{1.7857e-05Hz_0.00077secs}
        \label{fig:10X:cmca}
    }
    \subfloat[Firmware-based Logging Impacts]{
        \includegraphics[width=0.3\textwidth]{1.7857e-05Hz_0.13secs}
        \label{fig:10X:firmware}
    }}
   \caption{\textbf{Performance impacts of correctable errors using $10\times$ the current
                    correctable error rate from \cielo.  Three scenarios are shown:
                    hardware only correction with no logging ($150ns$ per event),
                    Software-based logging using the Corrected Machine Check
                    Architecture (CMCA) ($775\mu sec$ per event), and the
                    Firmware-based logging
                    using the Enhanced Machine Check Architecture (EMCA)
                    ($133msecs$ per event) }.
   }
\label{fig:10X}
\end{figure*}

\begin{figure*}
\centering{
    \subfloat[Hardware Correction Impacts]{
        \includegraphics[width=0.3\textwidth]{allHz_1.5e-7secs}
        \label{fig:100X:hardware}
    }
    \subfloat[Software-based Logging Impacts]{
        \includegraphics[width=0.3\textwidth]{0.00017857Hz_0.00077secs}
        \label{fig:100X:cmca}
    }
    \subfloat[Firmware-based Logging Impacts]{
        \includegraphics[width=0.3\textwidth]{0.00017857Hz_0.13secs}
        \label{fig:100X:firmware}
    }}
   \caption{\textbf{Performance impacts of correctable errors using $100\times$ the current
                    correctable error rate from \cielo.  Three scenarios are shown:
                    hardware only correction with no logging ($150ns$ per event),
                    Software-based logging using the Corrected Machine Check
                    Architecture (CMCA) ($775\mu sec$ per event), and the
                    Firmware-based logging
                    using the Enhanced Machine Check Architecture (EMCA)
                    ($133msecs$ per event) }.
   }
\label{fig:100X}
\end{figure*}

\begin{figure*}
\centering{
    \subfloat[Hardware Correction Impacts]{
        \includegraphics[width=0.3\textwidth]{allHz_1.5e-7secs}
        \label{fig:1000X:hardware}
    }
    \subfloat[Software-based Logging Impacts]{
        \includegraphics[width=0.3\textwidth]{0.0017857Hz_0.00077secs}
        \label{fig:1000X:cmca}
    }
    \subfloat[Firmware-based Logging Impacts]{
        \includegraphics[width=0.3\textwidth]{0.0017857Hz_0.13secs}
        \label{fig:1000X:firmware}
    }}
   \caption{\textbf{Performance impacts of correctable errors using $1000\times$ the current
                    correctable error rate from \cielo.  Three scenarios are shown:
                    hardware only correction with no logging ($150ns$ per event),
                    Software-based logging using the Corrected Machine Check
                    Architecture (CMCA) ($775\mu sec$ per event), and the
                    Firmware-based logging
                    using the Enhanced Machine Check Architecture (EMCA)
                    ($133msecs$ per event) }.
   }
\label{fig:1000X}
\end{figure*}


\subsection{Exploring Correctable Frequency and Duration}

\kbf{Add plots here ... }

\subsection{Discussion and Analysis}

\kbf{Add the applciation analysis stuff here}
