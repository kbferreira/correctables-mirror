
\section{Experiential Approach}
\label{sec:approach}

\subsection{APEI Injection Framework}

\subsection{Simulating Correctable Overheads}

In general, the communication structure of Message Passing Interface (MPI)
programs cannot be determined offline because message matches cannot be
established statically~\cite{bronevetsky2009communication}.  This makes
modeling application performance analytically challenging even if all
parameters of the application (e.g., the complete communication structure and
all relative inter-process timings) are known.  We therefore use a validated
discrete-event simulation framework to evaluate the impact of local
correctable error mitigation activities on the performance of real applications.
%for real applications via their message traces.

Our simulation-based approach models correctable error mitigation activities as CPU detours: periods
of time during which the CPU is taken from the application and used to compute and commit
checkpoint data.  This approach allows a level of
fidelity and control not always possible in implementation-based approaches. It
also allows us to examine simulated systems much larger than those generally available.

Our simulation framework is based on the freely available
\LogGOPSim~\cite{Hoefler:2010:LogGOPSim} and the tool chain provided  by Levy et
al.~\cite{Levy2013UsingSimulation}.  \LogGOPSim uses the LogGOPS model, an
extension of the well-known LogP model~\cite{Culler:1993:LogP}, to account for
the temporal cost of communication events.  An application's communication
events are generated from traces of the application's execution.  These traces
contain the sequence of MPI operations invoked by each application process.
\LogGOPSim uses these traces to reproduce all communication dependencies,
including indirect dependencies between processes which do not communicate
directly.

\LogGOPSim can also extrapolate traces from small application runs; a trace
collected by running the application with $p$ processes can be extrapolated to
simulate performance of the application running with $k\cdot p$ processes. The
extrapolation produces exact communication patterns for MPI collective
operations and approximates point-to-point
communications~\cite{Hoefler:2010:LogGOPSim}.  The validation of \LogGOPSim and
its trace extrapolation features have been documented
previously~\cite{Hoefler:2010:LogGOPSim}, along with the simulators ability to
accurately predict application performance in the presence of performance
perturbations~\cite{Ferreira:2014:Understanding,Levy2013UsingSimulation,Hoefler:2010:Characterizing}

\subsection{Simulating Applications with Correctable Errors}

\kbf{Add application workloads}
