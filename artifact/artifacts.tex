
\clearpage
\appendix
\section{Artifact Description: \kbf{add title}}
\label{sec:appendix}


\kbf{This is incorrect, a completely new folktale needs to be written}

%%%%%%%%%%%%%%%%%%%%%%%%%%%%%%%%%%%%%%%%%%%%%%%%%%%%%%%%%%%%%%%%%%%%%
\subsection{Abstract}

This artifact contains instructions on accessing the \LogGOPSim simulator and
the application execution traces that were used for the correctable DRAM error 
logging overhead experiments in \refsec{sec:results:correctable_dram_perf}. 
It also contains instructions on obtaining and using the failure log analysis
scripts used throughout \refsec{sec:results}.

%%%%%%%%%%%%%%%%%%%%%%%%%%%%%%%%%%%%%%%%%%%%%%%%%%%%%%%%%%%%%%%%%%%%%
\subsection{Description}

\subsubsection{Check-list (artifact meta information)}

{\small
\begin{itemize}
  \item {\bf Program}: C++ for \LogGOPSim, Python for log analysis
  \item {\bf Compilation}: g++ 4.2+ for \LogGOPSim, Python 2.7+
  \item {\bf Data set}: In this paper, we analyzed a corpus of failure data collected over
  the five-year lifetime of \cielo.  This corpus of data consists primarily of the contents
  of system logs.  A detailed description of the contents of these logs and the relevant 
  data that each contains is presented in \refsec{sec:method:system_logs} in the body of the paper.

  We also used a collection of application execution traces in our correctable DRAM error 
  experiments.  The applications used were: LAMMPS-lj, LAMMPS-crack, LAMMPS-SNAP,
  LULESH, miniFE, GTC-p, and CTH.  In addition, we also used execution traces for three 
  microbenchmarks: allreduce, reduce, and stencil.
  \item {\bf Run-time environment}: MacOS 10.8+ or any modern Linux distribution
  \item {\bf Hardware}: Any
  \item {\bf Experiment customization}: modify Makefile to specify different application
  traces or reliability logs
  \item {\bf Publicly available?}: All but one of the execution traces (CTH) is publicly 
  available.  Because CTH is an export-controlled application, its traces cannot be 
  released publicly.  

  System logs collected on \cielo contain significant proprietary information that is 
  covered by one or more non-disclosure agreements~(NDA).  As a result, we cannot publicly
  release this corpus of data.  Our repository contains a sample of random data that can
  be used to demonstrate the operation of our analysis scripts.
\end{itemize}
}

\subsubsection{How software can be obtained (if available)}
\begin{itemize}
\item \LogGOPSim is available at:\\
    \url{https://htor.inf.ethz.ch/research/LogGOPSim/LogGOPSim-1.1.tgz}.  

\item Failure log analysis scripts can be found in our repository at: 
    \detail{made up repo}
\end{itemize}

\subsubsection{Hardware dependencies}

\LogGOPSim will run on any modern general-purpose computer. However, the
largest application scale and duration that can be simulated depends on
available RAM.  Memory usage increases linearly in both and scale and application
runtime.

\subsubsection{Software dependencies}

Application simulations were run on a supercomputer running Linux located at
\detail{Sandia National Laboratories}.  Both \LogGOPSim and the log
analysis scripts have been tested on Linux and MacOS running Python 2.7,
Matplotlib 2.0.0, and g++ 4.2.1

\subsubsection{Datasets}

\begin{table}
\centering
\begin{tabular}{ l c }
\toprule
LogGOPS parameter & Value\\
\midrule
\textcolor{red}{L}atency                & 1.8$\mu s$ \\
\textcolor{red}{o}verhead per message   & 12.4$\mu s$ \\
\textcolor{red}{g}ap per message        & 2.6$\mu s$  \\
\textcolor{red}{G}ap per byte           & 1$ns$     \\
\textcolor{red}{O}verhead per byte      & 0$ns$     \\
\textcolor{red}{S}: rendezvous threshold  & 65,536 bytes \\
\bottomrule
\end{tabular}
\caption{
  LogGOPS parameters used in our study. Values measured using~\cite{netgauge-web}
}
\label{tab:logp}
\end{table}

Application traces used (excluding CTH) are located in the {\texttt{traces}} 
directory of our repository.  LogP parameters used are specified in \reftab{tab:logp}.

%%%%%%%%%%%%%%%%%%%%%%%%%%%%%%%%%%%%%%%%%%%%%%%%%%%%%%%%%%%%%%%%%%%%%
\subsection{Installation}

Installation instructions are available at: \url{https://htor.inf.ethz.ch/research/LogGOPSim/}

Log analysis install instructions:

{\texttt{\$ git clone \detail{A very fake repo}}}

%%%%%%%%%%%%%%%%%%%%%%%%%%%%%%%%%%%%%%%%%%%%%%%%%%%%%%%%%%%%%%%%%%%%%
\subsection{Experiment workflow}

{\texttt{\$ make DRAM-overheads \# For \LogGOPSim runs and correctable DRAM plots}}

{\texttt{\$ make analysis \# For log analysis plots}}
%%%%%%%%%%%%%%%%%%%%%%%%%%%%%%%%%%%%%%%%%%%%%%%%%%%%%%%%%%%%%%%%%%%%%
\subsection{Evaluation and expected result}

All generated figures will be placed in the {\texttt{figs}} directory

%%%%%%%%%%%%%%%%%%%%%%%%%%%%%%%%%%%%%%%%%%%%%%%%%%%%%%%%%%%%%%%%%%%%%
\subsection{Experiment customization}

New application traces can be added for analysis by adding the new trace
directory to the {\texttt{TRACES}} variable in the top-level Makefile

New reliability logs can be added by modifying the {\texttt{LOG\_LOCATION}}
variable in that same Makefile

%%%%%%%%%%%%%%%%%%%%%%%%%%%%%%%%%%%%%%%%%%%%%%%%%%%%%%%%%%%%%%%%%%%%%
\subsection{Notes}

Using \LogGOPSim to simulate the execution of an application running on 
a large number of nodes can be very time-consuming.  In extreme cases,
it may take a day or more to simulate a few minutes of execution of
very large applications.
