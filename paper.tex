% R&A Tracking number: 
\documentclass[conference]{IEEEtran}
\usepackage{url}
\usepackage{amsmath}
\usepackage{caption}
\usepackage[caption=false,font=footnotesize]{subfig}
\usepackage{graphicx}
\usepackage{verbatim}
\usepackage{xcolor}
\usepackage{xspace}
\usepackage{booktabs}
\usepackage{multirow}
\usepackage{grffile}
\usepackage{tikz}
\usepackage{cleveref}
\usepackage[Blind]{optional}
% Prevent single column figures from being rendered before double column figures
\usepackage{fixltx2e}
\usepackage{algorithm}
\usepackage{algpseudocode}

\graphicspath{{figs/}}

\opt{Blind}{
        \newcommand{\blake}{{\bf{[production system name]}}\xspace}
        \newcommand{\cielo}{{\bf{[developemt system name]}}\xspace}
        \newcommand{\detail}[1]{{\bf{[detail removed for double-blind review]}\xspace}}
}

\opt{NotBlind}{
        \newcommand{\blake}{Blake\xspace}
        \newcommand{\cielo}{Cielo\xspace}
        \newcommand{\detail}[1]{#1}
}

\newcommand{\sll}[1]{\textcolor{blue}{[sll: #1]}}
\newcommand{\kbf}[1]{\textcolor{red}{[kbf: #1]}}

%
% Some common definitions
%

\providecommand{\email}[1]{\url{#1}}
\newcommand{\LogGOPSim}{{\texttt{Log\-GOP\-Sim}}\xspace}
\newcommand{\selfish}{{\texttt{sel\-fish}}\xspace}

%
% ------------------------------------------------------------------------------
% Help functions for references
%
\providecommand{\assignmentname}{Exercise}
\providecommand{\equationname}{Equation}
\providecommand{\figurename}{Figure}
\providecommand{\figuresname}{Figures}
\providecommand{\tablename}{Table}
\providecommand{\nameofsection}{Section}
\providecommand{\nameofsections}{Sections}
\newcommand{\reffig}[1]{\figurename~\ref{#1}}
\newcommand{\refsubfig}[1]{\figurename~\subref{#1}}
\newcommand{\reftwosubfig}[2]{\figuresname~\subref{#1} and \subref{#2}}
\newcommand{\reftwofig}[2]{\figuresname~\ref{#1} and \ref{#2}}
\newcommand{\refthreesubfig}[3]{\figuresname~\subref{#1}, \subref{#2}, and \subref{#3}}
\newcommand{\refthreefig}[3]{\figuresname~\ref{#1}, \ref{#2}, and \ref{#3}}
\newcommand{\reftab}[1]{\tablename~\ref{#1}}
\newcommand{\reflst}[1]{\lstlistingname~\ref{#1}}
\newcommand{\refchap}[1]{\chaptername~\ref{#1}}
\newcommand{\refapp}[1]{\appendixname~\ref{#1}}
%\newcommand{\refsec}[1]{\sectionname~\ref{#1}}
\newcommand{\refsec}[1]{\nameofsection~\ref{#1}}
\newcommand{\reftwosec}[2]{\nameofsection~\ref{#1} and \ref{#2}}
\newcommand{\refeq}[1]{\equationname~\ref{#1}}
\newcommand{\refpage}[1]{on page~\pageref{#1}}


\newlength{\fullfigWidth}
\setlength{\fullfigWidth}{0.855\columnwidth}

\newlength{\twoacross}
\setlength{\twoacross}{0.4\textwidth}

\newlength{\threeacross}
\setlength{\threeacross}{0.315\textwidth}

%
% ------------------------------------------------------------------------------
% Help functions for editing
%
\definecolor{NoteColor}{rgb}{0.71, 0.22, 0.21}
\newcommand{\note}[1]{{\color{NoteColor}\bf Note: }{\color{NoteColor} #1}}
\newcommand{\hl}[1]{\textcolor{red}{\texttt{#1}}}



%
% ------------------------------------------------------------------------------
% General
%
\newcommand*{\OS}{operating system\xspace}
\newcommand*{\OSes}{operating systems\xspace}
\newcommand{\rmpi}{\textit{r}MPI\xspace}



%
% Programming
%
\newcommand*{\file}[1]{{\small\texttt{#1}}\xspace}
\newcommand*{\cmd}[1]{{\small\texttt{#1}}\xspace}
\newcommand*{\var}[1]{{\small\texttt{#1}}\xspace}
\newcommand*{\keyword}[1]{{\small\texttt{#1}}\xspace}
% \newcommand*{\func}[1]{\texttt{\textcolor{funcfg}{#1()}}\xspace}
\newcommand*{\func}[1]{{\small\texttt{#1()}}\xspace}
% \newcommand*{\mfunc}[1]{\texttt{\textcolor{funcfg}{#1()}}\index{#1 (func)}\index{MPI functions!#1}\xspace}
\newcommand*{\mfunc}[1]{{\small\texttt{#1()}}\index{#1 (func)}\index{MPI functions!#1}\xspace}
\newcommand*{\pfunc}[1]{{\small\texttt{#1()}}\index{#1 (func)}\index{Pthreads functions!#1}\xspace}
\newcommand*{\code}[1]{{\small\texttt{#1}}\xspace}
\renewcommand*{\arg}[1]{{\small\texttt{#1}}\xspace}
\newcommand*{\type}[1]{{\small\texttt{#1}}\index{#1 (type)}\index{data types!#1}\xspace}
\newcommand*{\mtype}[1]{{\small\texttt{#1}}\index{#1 (type)}\index{MPI data types!#1}\xspace}
\newcommand*{\ptype}[1]{{\small\texttt{#1}}\index{#1 (type)}\index{pthreads data types!#1}\xspace}
\newcommand*{\mconst}[1]{{\small\texttt{#1}}\index{#1 (const.)}\index{MPI constants!#1}\xspace}
\newcommand*{\pconst}[1]{{\small\texttt{#1}}\index{#1 (const.)}\index{pthreads constants!#1}\xspace}

\newcommand*{\MAXINT}{\type{MAX\_INT}\xspace}
\newcommand*{\NULL}{\type{NULL}\xspace}




%
% MPI types
%
\newcommand*{\MPIStatus}{\mtype{MPI\_\-Status}}
\newcommand*{\MPISOURCE}{\mtype{MPI\_\-SOURCE}}
\newcommand*{\MPITAG}{\mtype{MPI\_\-TAG}}



%
% MPI data types
%
\newcommand*{\MPICHAR}{\mtype{MPI\_\-CHAR}}
\newcommand*{\MPISIGNEDCHAR}{\mtype{MPI\_\-SIGNED\_\-CHAR}}
\newcommand*{\MPISHORT}{\mtype{MPI\_\-SHORT}}
\newcommand*{\MPIINT}{\mtype{MPI\_\-INT}}
\newcommand*{\MPIINTEGER}{\mtype{MPI\_\-INTEGER}} % Fortran only
\newcommand*{\MPILONG}{\mtype{MPI\_\-LONG}}
\newcommand*{\MPILONGLONGINT}{\mtype{MPI\_\-LONG\_\-LONG\_\-INT}}
\newcommand*{\MPILONGLONG}{\mtype{MPI\_\-LONG\_\-LONG}} % Exists due to an error in MPI-2.0. Should be MPI_LONG_LONG_INT
\newcommand*{\MPIUNSIGNEDLONGLONG}{\mtype{MPI\_\-UNSIGNED\_\-LONG\_\-LONG}}
\newcommand*{\MPIUNSIGNEDCHAR}{\mtype{MPI\_\-UNSIGNED\_\-CHAR}}
\newcommand*{\MPIUNSIGNEDSHORT}{\mtype{MPI\_\-UNSIGNED\_\-SHORT}}
\newcommand*{\MPIUNSIGNED}{\mtype{MPI\_\-UNSIGNED}}
\newcommand*{\MPIUNSIGNEDLONG}{\mtype{MPI\_\-UNSIGNED\_\-LONG}}
\newcommand*{\MPIFLOAT}{\mtype{MPI\_\-FLOAT}}
\newcommand*{\MPIDOUBLE}{\mtype{MPI\_\-DOUBLE}}
\newcommand*{\MPILONGDOUBLE}{\mtype{MPI\_\-LONG\_\-DOUBLE}}
\newcommand*{\MPIBYTE}{\mtype{MPI\_\-BYTE}}
\newcommand*{\MPIPACKED}{\mtype{MPI\_\-PACKED}}
\newcommand*{\MPIWCHAR}{\mtype{MPI\_\-WCHAR}}



%
% MPI Constants
%
\newcommand*{\MPIINPLACE}{\mconst{MPI\_\-IN\_\-PLACE}}
\newcommand*{\MPICOMMWORLD}{\mconst{MPI\_\-COMM\_\-WORLD}}
\newcommand*{\MPIANYSOURCE}{\mconst{MPI\_\-ANY\_\-SOURCE}}
\newcommand*{\MPIANYTAG}{\mconst{MPI\_\-ANY\_\-TAG}}
\newcommand*{\MPITAGUB}{\mconst{MPI\_\-TAG\_\-UB}}
\newcommand*{\MPISUCCESS}{\mconst{MPI\_\-SUCCESS}}
\newcommand*{\MPISTATUSIGNORE}{\mconst{MPI\_\-STATUS\_\-IGNORE}}
\newcommand*{\MPISTATUSESIGNORE}{\mconst{MPI\_\-STATUSES\_\-IGNORE}}
\newcommand*{\MPIWTIMEISGLOBAL}{\mconst{MPI\_\-WTIME\_\-IS\_\-GLOBAL}}
\newcommand*{\MPIUNDEFINED}{\mconst{MPI\_\-UNDEFINED}}
\newcommand*{\MPIREQUESTNULL}{\mconst{MPI\_\-REQUEST\_\-NULL}}



%
% MPI Collective Functions
%
\newcommand*{\MPIBarrier}{\mfunc{MPI\_\-Barrier}}
\newcommand*{\MPIBcast}{\mfunc{MPI\_\-Bcast}}
\newcommand*{\MPIReduce}{\mfunc{MPI\_\-Reduce}}
\newcommand*{\MPIGather}{\mfunc{MPI\_\-Gather}}
\newcommand*{\MPIScatter}{\mfunc{MPI\_\-Scatter}}
\newcommand*{\MPIAllgather}{\mfunc{MPI\_\-Allgather}}
\newcommand*{\MPIAllreduce}{\mfunc{MPI\_\-Allreduce}}
\newcommand*{\MPIAlltoall}{\mfunc{MPI\_\-Alltoall}}
\newcommand*{\MPIScan}{\mfunc{MPI\_\-Scan}}
\newcommand*{\MPIExscan}{\mfunc{MPI\_\-Exscan}}
\newcommand*{\MPIGatherv}{\mfunc{MPI\_\-Gatherv}}
\newcommand*{\MPIScatterv}{\mfunc{MPI\_\-Scatterv}}
\newcommand*{\MPIAllgatherv}{\mfunc{MPI\_\-Allgatherv}}
\newcommand*{\MPIAlltoallv}{\mfunc{MPI\_\-Alltoallv}}
\newcommand*{\MPIReducescatter}{\mfunc{MPI\_\-Reduce\_\-scatter}}
\newcommand*{\MPIAlltoallw}{\mfunc{MPI\_\-Alltoallw}}
\newcommand*{\MPIOpcreate}{\mfunc{MPI\_\-Op\_\-create}}
\newcommand*{\MPIOpfree}{\mfunc{MPI\_\-Op\_\-free}}


%
% MPI Functions
%
\newcommand*{\MPIInit}{\mfunc{MPI\_\-Init}}
\newcommand*{\MPIInitalized}{\mfunc{MPI\_\-Initialized}}
\newcommand*{\MPIFinalize}{\mfunc{MPI\_\-Finalize}}
\newcommand*{\MPICommrank}{\mfunc{MPI\_\-Comm\_\-rank}}
\newcommand*{\MPICommsize}{\mfunc{MPI\_\-Comm\_\-size}}
\newcommand*{\MPICommdup}{\mfunc{MPI\_\-Comm\_\-dup}}
\newcommand*{\MPICommsplit}{\mfunc{MPI\_\-Comm\_\-split}}
\newcommand*{\MPIRequestfree}{\mfunc{MPI\_\-Request\_\-free}}
\newcommand*{\MPIWtime}{\mfunc{MPI\_\-Wtime}}
\newcommand*{\MPISend}{\mfunc{MPI\_\-Send}}
\newcommand*{\MPIIsend}{\mfunc{MPI\_\-Isend}}
\newcommand*{\MPIRecv}{\mfunc{MPI\_\-Recv}}
\newcommand*{\MPIIrecv}{\mfunc{MPI\_\-Irecv}}
\newcommand*{\MPIWait}{\mfunc{MPI\_\-Wait}}
\newcommand*{\MPIWaitall}{\mfunc{MPI\_\-Wait\-all}}
\newcommand*{\MPIWaitany}{\mfunc{MPI\_\-Wait\-any}}
\newcommand*{\MPITest}{\mfunc{MPI\_\-Test}}
\newcommand*{\MPITestall}{\mfunc{MPI\_\-Test\-all}}
\newcommand*{\MPITestany}{\mfunc{MPI\_\-Test\-any}}
\newcommand*{\MPIProbe}{\mfunc{MPI\_\-Probe}}
\newcommand*{\MPIIprobe}{\mfunc{MPI\_\-Iprobe}}
\newcommand*{\MPIAbort}{\mfunc{MPI\_\-Abort}}
\newcommand*{\MPICancel}{\mfunc{MPI\_\-Cancel}}
\newcommand*{\MPICartcreate}{\mfunc{MPI\_\-Cart\_\-create}}
\newcommand*{\MPIStart}{\mfunc{MPI\_\-Start}}
\newcommand*{\MPIRecvinit}{\mfunc{MPI\_\-Recv\_\-init}}
\newcommand*{\MPIStartall}{\mfunc{MPI\_\-Startall}}
\newcommand*{\MPIGrouprank}{\mfunc{MPI\_\-Group\_\-rank}}
\newcommand*{\MPIGroupstar}{\mfunc{MPI\_\-Group\_$\ast$}}



%
% MPI Data Type Functions
%
\newcommand*{\MPITypecontiguous}{\mfunc{MPI\_\-Type\_\-contiguous}}
\newcommand*{\MPITypecommit}{\mfunc{MPI\_\-Type\_\-commit}}
\newcommand*{\MPITypefree}{\mfunc{MPI\_\-Type\_\-free}}
\newcommand*{\MPITypevector}{\mfunc{MPI\_\-Type\_\-vector}}
\newcommand*{\MPITypeindexed}{\mfunc{MPI\_\-Type\_\-indexed}}
\newcommand*{\MPITypegetextent}{\mfunc{MPI\_\-Type\_\-get\_\-extent}}
\newcommand*{\MPITypeextent}{\mfunc{MPI\_\-Type\_\-extent}} % MPI-1 depreciated
\newcommand*{\MPITypelb}{\mfunc{MPI\_\-Type\_\-lb}} % MPI-1 depreciated
\newcommand*{\MPITypeub}{\mfunc{MPI\_\-Type\_\-ub}} % MPI-1 depreciated
\newcommand*{\MPIGetcount}{\mfunc{MPI\_\-Get\_\-count}}
\newcommand*{\MPITypegettrueextent}{\mfunc{MPI\_\-Type\_\-get\_\-true\_\-extent}}
\newcommand*{\MPITypesize}{\mfunc{MPI\_\-Type\_\-size}}
\newcommand*{\MPITypecreateresized}{\mfunc{MPI\_\-Type\_\-create\_\-resized}}
\newcommand*{\MPITypecreatestruct}{\mfunc{MPI\_\-Type\_\-create\_\-struct}}
\newcommand*{\MPIGetaddress}{\mfunc{MPI\_\-Get\_\-address}}




% MPI ops
\newcommand*{\MPIMAX}{\code{MPI\_\-MAX\index{MPI\_\-MAX}\index{MPI predefined functions!MPI\_\-MAX}}}
\newcommand*{\MPIMIN}{\code{MPI\_\-MIN\index{MPI\_\-MIN}\index{MPI predefined functions!MPI\_\-MIN}}}
\newcommand*{\MPISUM}{\code{MPI\_\-SUM\index{MPI\_\-SUM}\index{MPI predefined functions!MPI\_\-SUM}}}
\newcommand*{\MPIPROD}{\code{MPI\_\-PROD\index{MPI\_\-PROD}\index{MPI predefined functions!MPI\_\-PROD}}}
\newcommand*{\MPILAND}{\code{MPI\_\-LAND\index{MPI\_\-LAND}\index{MPI predefined functions!MPI\_\-LAND}}}
\newcommand*{\MPIBAND}{\code{MPI\_\-BAND\index{MPI\_\-BAND}\index{MPI predefined functions!MPI\_\-BAND}}}
\newcommand*{\MPILOR}{\code{MPI\_\-LOR\index{MPI\_\-LOR}\index{MPI predefined functions!MPI\_\-LOR}}}
\newcommand*{\MPIBOR}{\code{MPI\_\-BOR\index{MPI\_\-BOR}\index{MPI predefined functions!MPI\_\-BOR}}}
\newcommand*{\MPILXOR}{\code{MPI\_\-LXOR\index{MPI\_\-LXOR}\index{MPI predefined functions!MPI\_\-LXOR}}}
\newcommand*{\MPIBXOR}{\code{MPI\_\-BXOR\index{MPI\_\-BXOR}\index{MPI predefined functions!MPI\_\-BXOR}}}
\newcommand*{\MPIMAXLOC}{\code{MPI\_\-MAXLOC\index{MPI\_\-MAXLOC}\index{MPI predefined functions!MPI\_\-MAXLOC}}}
\newcommand*{\MPIMINLOC}{\code{MPI\_\-MINLOC\index{MPI\_\-MINLOC}\index{MPI predefined functions!MPI\_\-MINLOC}}}

\newcommand*{\MPIFLOATINT}{\code{MPI\_\-FLOAT\_INT\index{MPI\_\-FLOAT\_\-INT}\index{MPI predefined datatypes!MPI\_\-FLOAT\_\-INT}}}
\newcommand*{\MPIDOUBLEINT}{\code{MPI\_\-DOUBLE\_\-INT\index{MPI\_\-DOUBLE\_\-INT}\index{MPI predefined datatypes!MPI\_\-DOUBLE\_\-INT}}}
\newcommand*{\MPILONGINT}{\code{MPI\_\-LONG\_\-INT\index{MPI\_\-LONG\_\-INT}\index{MPI predefined datatypes!MPI\_\-LONG\_\-INT}}}
\newcommand*{\MPITWOINT}{\code{MPI\_\-2INT\index{MPI\_\-2INT}\index{MPI predefined datatypes!MPI\_\-2INT}}}
\newcommand*{\MPISHORTINT}{\code{MPI\_\-SHORT\_\-INT\index{MPI\_\-SHORT\_\-INT}\index{MPI predefined datatypes!MPI\_\-SHORT\_\-INT}}}
\newcommand*{\MPILONGDOUBLEINT}{\code{MPI\_\-LONG\_\-DOUBLE\_\-INT\index{MPI\_\-LONG\_\-DOUBLE\_\-INT}\index{MPI predefined datatypes!MPI\_\-LONG\_\-DOUBLE\_\-INT}}}

\hyphenation{op-tical net-works semi-conduc-tor}


\begin{document}
\title{Understanding the Effects of Correctable Errors at Scale}

\opt{NotBlind}{
\author{\IEEEauthorblockN{
Kurt B. Ferreira\IEEEauthorrefmark{1},
Scott Levy\IEEEauthorrefmark{1},
Nathan DeBardeleben\IEEEauthorrefmark{2},
Elisabeth Baseman\IEEEauthorrefmark{2},
Taniya Siddiqua\IEEEauthorrefmark{3},
Vilas Sridharan\IEEEauthorrefmark{3}, \\
Victor Kuhns\IEEEauthorrefmark{1},
}
\IEEEauthorblockA{\IEEEauthorrefmark{1}Center for Computing Research, Sandia National Laboratories\\
\email{{sllevy, kbferre, ktpedre,vgkuhns}@sandia.gov}}
\IEEEauthorblockA{\IEEEauthorrefmark{2}
Ultrascale Systems Research Center, Los Alamos National Laboratory\\
\email{{ndebard,lissa}@lanl.gov}}
\IEEEauthorblockA{\IEEEauthorrefmark{3}
RAS Architecture, Advanced Micro Devices, Inc.\\
\email{{taniya.siddiqua, vilas.sridharan}@amd.com}}
} % end AUTHOR block
}

\opt{Blind}{
        \author{}
}

\maketitle

\begin{abstract}

% Motivation
        Fault-tolerance poses a major challenge for future large-scale systems.
        Active research in the field typically focuses on mitigating the
        effects of uncorrectable errors, those fatal errors that typically
        require an application to restart from a known good state.
% Methods/Procedures
        However, the impacts of the most common error of large-scale systems,
        correctable errors, is typically overlooked.  Moreover, increased
        memory volumes and expected technology changes on future extreme-scale
        system may make these errors even more likely and more of a concern.
        In this work, we use a simulation-based approach to show how local
        correctable errors can significantly affect the performance of key
        extreme-scale workloads.
% Findings
        Our study shows that local delays due to correctable errors can
        propagate through MPI message synchronization to other processes,
        causing a cascading series of delays.  We also find that though much of
        the focus on correctable errors is focused on reducing failure rates,
        reducing the rate of each individual error may be more impacting on
        overheads at scale. Finally,  this study outlines the errors
        frequencies, durations, and scales in which performance is
        significantly impacted for a number of key extreme-scale workloads.
% Conclusion/Impacts
        This work provides critical analysis and insight into the overheads of
        common correctable errors and provides practical advice to users and
        systems administrators in an effort to fine-tune performance to
        application and system characteristics.  \end{abstract}




\section{Introduction}
\label{sec:intro}

Maintaining the performance of high-performance computing~(HPC) applications as
failures become more and more frequent is a major challenge that needs to be
addressed for next-generation extreme-scale systems.  Many recent studies have
demonstrated that hardware failures are expected to become ever more
common~\cite{Bergman08exascalecomputing}.  Increasing the scale of HPC systems
requires the aggregation of larger number of individual components.  More
components means more frequent failures.  Current systems use powerful
error-correcting codes~(ECC), e.g., chipkill-correct, to protect against DRAM
errors.  However, chipkill-correct (and other similar techniques) require the
activation of a large number of memory devices (four times more than
less-protective techniques like single error correct double error
detect~(SECDED))~\cite{Jian13}.  Activating more memory devices requires more
power for each memory access.  However, because of tightening power budgets on
next-generation systems~\cite{Bergman08exascalecomputing}, it is not yet clear
that chipkill-correct will continue to be viable.  Reduced device-feature sizes
also have the potential to result in more frequent failures.  Understanding the
implications of these trends requires that we have detailed knowledge of how
failures affect current leadership-class systems.

Error detection and handling is critical to pinpointing failing components and
taking corrective action in a timely fashion. This error handling is typically a
cooperative activity between the platform hardware, firmware (UEFI or BIOS), and
the host operating system.  Error instances are signaled in host firmware or
the OS directly. The firmware will read the hardware registers and analyze the
component that generated the error in an effort to assess the severity of the
error. Firmware will then create a detailed description of the error and notify
the OS of its occurrence. Host firmware may additionally communicate this error
to baseboard management controllers for system management purposes.  When the
error notification is signaled to the OS, either directly or from host
firmware, the OS typically inspect hardware registers or firmware and initiate
corrective action.  This handling by the OS and/or firmware can perturb
application progress, even in cases where the failure does not initiate a
restart.

Errors are typically classified into three categories: Correctable,
Uncorrectable, and Fatal. Correctable errors (CE) are errors that can be
corrected or recovered in hardware such that platform state is left as if no
error had occurred.  An example of a CE is is a single bit (or single symbol in
the case of chipkill) error.  Uncorrectable errors (UE) are those that could be
detected by hardware, but could not be corrected.  Multi-bit/Multi-symbol errors
are an example of such an error.  In many cases it is possible for the system to
continue functioning in the face of these errors, perhaps with a certain amount
of lost state.  Finally, fatal errors are those errors such that continuing
after this error would make the system unreliable.  Fatal errors typically
initiate a full system halt.

Recent works typically focus on uncorrectable and fatal errors, those errors that cause
applications to restart.  However, the impacts of the most common error of
large-scale systems, correctable errors, is typically overlooked. An analysis of
failures on recent leadership-class systems show that the correctable failure
rate is a factor \hl{ADD} more frequent than uncorrectable errors.  Correctable
errors are typically handled at a hardware level, and not reflected to the
application.  While the application continues to make progress in the presence
of these correctable errors and does not need to restart,the mechanism to
correct and log these errors have the potential to impact application
performance by delaying application computation. In this work we seek to address
the question of the performance impacts of correctable errors

\let\workinterval\relax
\let\ckpttime\relax
\let\txdelay\relax
\let\msgtime\relax
\let\minheight\relax
\newcommand{\workinterval}{1.25cm}
\newcommand{\ckpttime}{0.625cm}
\newcommand{\txdelay}{2.0mm}
\newcommand{\msgtime}{\workinterval+\txdelay}
\newcommand{\minheight}{0.5cm}
\usetikzlibrary{positioning}

\tikzstyle{position}=[fill=none,text=white,draw=none]
\tikzstyle{proc}=[fill=none,text=black,draw=none,shape=rectangle]
\tikzstyle{origtotal}=[fill=none,text=black,draw=black,shape=rectangle,
                       minimum width=3*\workinterval+2*\txdelay,
                       minimum height=\minheight]
\tikzstyle{coordtotal}=[fill=none,text=black,draw=black,shape=rectangle,
                        minimum width=3*\workinterval+\ckpttime+2*\txdelay,
                        minimum height=\minheight]
\tikzstyle{uncoordtotal}=[fill=none,text=black,draw=black,shape=rectangle,
                          minimum width=3*\workinterval+2*\ckpttime+2*\txdelay,
                          minimum height=\minheight]
\tikzstyle{ckpt}=[fill=black,text=white,draw=none,shape=rectangle,
                  minimum width=\ckpttime,minimum height=\minheight]
\tikzstyle{stall}=[fill=black!20,text=white,draw=black,shape=rectangle,
                   minimum height=\minheight]

\begin{figure*}[bt]
\centering{
\subfloat[without CE activity]{
\resizebox{0.25\textwidth}{!}{
\begin{tikzpicture}[semithick]

\node [proc] (p0) {$p_0$};
\node [position, right= 0.125cm of p0, minimum width=3*\workinterval+2*\ckpttime+2*\txdelay] (n0) {};
\node [origtotal, right= 0.125cm of p0]  (v1) {};
\node [position, right= \msgtime of v1.north west ]  (v2) {};
\node [position, right= \msgtime of v2.west ]  (v4) {};

\node [proc, below = 5mm of p0] (p1) {$p_1$};
\node [origtotal, right= 0.125cm of p1]  (v5) {};
\node [position, right= \msgtime of v5.west ]  (v6) {};
\node [position, right= \workinterval of v6.south west ]  (v8) {};

\node [proc, below of=p1] (p2) {$p_2$};
\node [origtotal, right= 0.125cm of p2]  (v9) {};
\node [position, right= \msgtime of v9.west ]  (v10) {};
\node [position, right= \msgtime of v10.west ]  (v11) {};

\draw [->, thick] (v1.south west) ++(\workinterval,0) -- ++(\txdelay, -5.00mm) 
      node[above right] {$\mathbf{m}_1$};
\draw [->, thick] (v8.south west) -- ++(\txdelay, -5.00mm) node[above right] {$\mathbf{m}_2$};

\draw [thick, dashed] (v2.north west) ++(0.0cm, 5.0mm) -- (v10.south west) -- ++(0, -5.0mm) node[below] {$t_1$};
\draw [thick, dashed] (v4.north west) ++(0.0cm, 5.0mm) -- (v11.south west) -- ++(0, -5.0mm) node[below] {$t_2$};
\end{tikzpicture} 
}\label{subfig:no_ckpt}
}
%
%
%\subfigure[uncoordinated checkpointing]{
\subfloat[with local CE activity delays]{
\resizebox{0.25\textwidth}{!}{
\begin{tikzpicture}[semithick]

\node [proc] (p0) {$p_0$};
\node [uncoordtotal, right= 0.125cm of p0]  (v1) {};
\node [position, right= \msgtime+\ckpttime of v1.north west ]  (v2) {};
\node [ckpt, right= \workinterval of v1.west ]  (v3) {$\delta$};
\node [position, right= \msgtime+\ckpttime of v2.west ]  (v4) {};

\node [proc, below of=p0] (p1) {$p_1$};
\node [uncoordtotal, right= 0.125cm of p1]  (v5) {};
\node [position, right= \msgtime+\ckpttime of v5.west ]  (v6) {};
\node [stall, minimum width=\ckpttime, left= 0.0cm of v6.west ]  (v7) {};
\node [ckpt, right= 0.00cm of v6.west ]  (v7) {$\delta$};
\node [position, right= \workinterval+\ckpttime of v6.south west ]  (v8) {};

\node [proc, below of=p1] (p2) {$p_2$};
\node [uncoordtotal, right= 0.125cm of p2]  (v9) {};
\node [position, right= \msgtime+\ckpttime of v9.west ]  (v10) {};
\node [position, right= \msgtime+\ckpttime of v10.west ]  (v11) {};
\node [stall, minimum width=2*\ckpttime, left= 0.00cm of v11.west ]  (v12) {};

\draw [->, thick] (v1.south west) ++(\workinterval+\ckpttime,0) -- ++(\txdelay, -5.00mm) 
      node[above right] {$\mathbf{m}_1$};
\draw [->, thick] (v8.south west) -- ++(2.0mm, -5.00mm) node[above right] {$\mathbf{m}_2$};

\draw [thick, dashed] (v2.north west) ++(0.0cm, 5.0mm) -- (v10.south west) -- ++(0, -5.0mm) node[below] {$t_1+\delta$};
\draw [thick, dashed] (v4.north west) ++(0.0cm, 5.0mm) -- (v11.south west) -- ++(0, -5.0mm) node[below] {$t_2+2\delta$};
\end{tikzpicture} 
}\label{subfig:uncoord_ckpt}
}
}
\caption{
        Example of how delays introduced by local correctable error (CE)
        activities may propagate along application communication dependencies.
        The processes $p_1$, $p_2$, and $p_3$ exchange two messages $m_1$ and
        $m_2$ in each of the three scenarios. The black regions marked with a
        white $\delta$ denote the execution of CE mitigation activities.  The
        grey regions denote periods in which the execution of a process is
        stalled due to an unsatisfied communication dependency.
}\label{fig:propagation}
\end{figure*}


In an effort to better understand the impacts from correctable errors, in this
paper we present in-depth analyses that reveal how the interplay between
application and hardware/firmware/OS correctable activities impact performance.
We anticipate that this interplay will become increasingly more important as
node counts and memory volumes increase dramatically on future extreme-scale
systems, further increasing error rates.  Additionally, smaller feature sizes,
manufacturing variances, hardware aging effects, and sub-threshold logic have
the potential to further exacerbate this problem and increase the correctable
and uncorrectable error rates.  More specifically, using a wide array of
applications, we demonstrate that the overheads associated with  correctable
errors can have a significant impact on overall application performance.  These
local correctable errors introduce overheads that can amplified or absorbed
globally by an application depending on an application's communication
activities.  We use profiles of these communication activities as well as
microbenchmark studies to more precisely attribute each application's
performance to its communication operations.

The possibly of correctable error activities inducing delays amongst application
processes, including those that do not directly communicate, is analogous to the
manner in which operating system noise (or \emph{jitter}) can affect HPC
applications~\cite{Hoefler:2010:Characterizing, Ferreira:08:characterizing}.
\Cref{fig:propagation} illustrates this phenomenon.  \Cref{subfig:no_ckpt} shows
a simple application running across three processes ($p_0$, $p_1$, and $p_2$)
with no delays due to CE.  These three processes exchange two messages,
$\mathbf{m}_1$ and $\mathbf{m}_2$.  We assume here that these messages represent
strict dependencies: any delay in the arrival of a message requires the
recipient to stall until the message is received. \Cref{subfig:uncoord_ckpt}
illustrates the potential impacts due to local CE mitigation activity.  If $p_0$
receives a CE at the instant before it would have otherwise sent $\mathbf{m}_1$,
then $p_1$ is forced to wait (the waiting period is shown in grey) until the
message arrives.  If $p_1$ subsequently receives a CE before sending
$\mathbf{m}_2$, then $p_2$ is forced to wait.  Part of the time that $p_2$
spends waiting is due to a delay that was originated by $p_0$, which it does not
communicate with.  The key point is that delays due to correctable error
activities can propagate based on communication dependencies in the application.

Based on the studies presented in this paper, we make the following
contributions:

\begin{itemize}

\item we demonstrate the impacts of the hardware/firmware/OS activities
        associated with correctable errors, and how those activities can changes
        based on platform configuration \S{?};

\item we analyze the potential impacts of this correctable errors on application
        performance and how these impacts many change with increased errors rate
        \S\S{?};

\item we show how an application's scale and communication pattern can dictate
        whether local overheads due to correctables are amplifies or absorbed by
        other processes \S\S{?}; and

\item we show the interplay between the correctable error rate and the duration
        of each mitigation activity, providing prescriptive advice on how best
        to reduce overheads due to these common errors on leadership-class platforms
        \S{?}.

\end{itemize}

Overall, this work provides critical analysis and insight into the overheads of
common correctable errors and provides practical advice to users and systems
administrators in an effort to fine-tune performance to application and system
characteristics.



\section{Experimental Approach}
\label{sec:approach}

\subsection{APEI Error Injection}
To develop a detailed understanding of the cost of recovering from correctable 
errors, we collected empirical data using ACPI Platform Error Interfaces~(APEI) Error 
Injection, \emph{see} \refsec{sec:results:dram_correctable}.  APEI is part of the 
Advanced Configuration and Power Interface (ACPI).  ACPI is a standard that defines how 
operating systems interact with the hardware components that comprise the systems on 
which they run.  APEI defines an interface by which the operating system can be notified 
of errors.  The APEI also provides a mechanism for injecting hardware errors: the error 
injection table~(EINJ).

%we use the error injection table (EINJ) support provided as part
%these impacts we need a method of injecting these errors in a system to measure
%the costs.  A number of methods exist to inject these types of errors on a
%running system: from hardware-specific DRAM daughter cards that synthetically
%flip memory cells within the DIMM device to methods that stress the memory
%system in an effort to induce failures. 

%In this work, we use the error injection table (EINJ) support provided as part
%of the Advanced Configuration and Power Interface (ACPI)
%specification~\cite{ACPISpec}.  

EINJ provides a platform-independent interface that enables the operating system to
inject hardware errors to the platform without requiring platform-specific software 
support. The primary objective of this mechanism is to validate operating system 
Reliability, Availability, and Serviceability~(RAS) features.  The ACPI specification 
defines several error types for EINJ.  The supported errors are summarized in \Cref{tab:einj}.  
EINJ is not supported on every platform; it requires support from the host processor, 
OS, and firmware/BIOS.  Moreover, platforms that support EINJ may not support all 
possible error types.  For example, our test platform supports only the \emph{Memory Correctable} 
and \emph{Memory Uncorrectable} error types.

\begin{table}
\centering
\begin{tabular}{ c l }
\toprule
Error Type Value & Error Description \\
\midrule
 0x00000001 & Processor Correctable\\
 0x00000002 & Processor Uncorrectable non-fatal \\
 0x00000004 & Processor Uncorrectable fatal \\
        {\bf{0x00000008}} & {\bf{Memory Correctable}} \\
 0x00000010 & Memory Uncorrectable non-fatal \\
 0x00000020 & Memory Uncorrectable fatal \\
 0x00000040 & PCI Express Correctable \\
 0x00000080 & PCI Express Uncorrectable fatal \\
 0x00000100 & PCI Express Uncorrectable non-fatal \\
 0x00000200 & Platform Correctable \\
 0x00000400 & Platform Uncorrectable non-fatal \\
 0x00000800 & Platform Uncorrectable fatal \\
\bottomrule
\end{tabular}
\vspace{.6em}
\caption{
        Available error types defined in ACPI specification.  The highlighted value
        (\emph{Memory Correctable}) was to collect the data presented in this paper.
}
\label{tab:einj}
\end{table}

%On our test platform, only \emph{Memory
%Correctable and memory uncorrectable error types are supported.  
\sll{KBF: please make sure that I've got this right.}\\
Using the EINJ table to inject errors on Linux-based systems is accomplished by writing to virtual 
files in the sysfs.\footnote{All of the relevant virtual files are in the \texttt{/sys/kernel/debug/apei/einj}
directory.  Additional information about Linux EINJ support, including a simple example of injecting an
error, is available in the Linux kernel documentation, \emph{see} \cite{einj_web}.} Using this interface, 
the user can specify the error type and target memory address, and trigger the injection of the error.

\subsection{Memory Failure Logging}

\kbf{The following paragraph is nearly identical to parallel submission,
change}

DRAM on typical modern systems is protected by an error-correcting
codes~(ECC).  When the memory controller detects a memory error, it attempts to
use the ECC to correct the error.  If it is able to correct the error, the
error is recorded as a \emph{correctable error}~(CE).  If it is unable to
correct the error, the error is recorded as a \emph{detected, uncorrectable
error}~(DUE).  On x86-based processors, correctable errors are recorded in 
registers provided by the x86 Machine Check Architecture~(MCA)~\cite{AMD,IntelGuide}.  
These registers are polled periodically and their contents are used to record
detailed information about the occurrence of the error is written to the console log.
This information includes the physical address where the error occurred and ECC
syndrome data that describes the cause of the error.  Decoding the recorded
information about each error allows us to identify the physical location of
each logged error, but this decoding process takes time, perturbing application
performance.

For correctable errors, there are two types of processor notification: software-based 
and firmware-based.  In the case of software-based notification, a Corrected Machine Check Interrupt
(CMCI)~\cite{IntelGuide,Gottscho:2017:Measuring} is generated which records
the DRAM error and the time of its occurrence.  However, it may be difficult
to determine the precise DRAM location of an error because of complexities related
to memory organization~\cite{Gottscho:2017:Measuring}.  As a result, mitigating
failures with memory page retirement~\cite{Tang:2006:Assessment} may not always
be possible with CMCI-based reporting.  With firmware-based notification, the
information recorded when the error occurs includes the physical address and the
specific DRAM device where the error occurred.  Firmware-based notification relies on the
Enhanced Machine Check Architecture (EMCA)~\cite{MCAEnhancements}.  While this
method allows for precise identification of the source of the error, it is very
expensive.  It requires the system to enter System Management Mode~(SMM) which halts
all forward progress on \emph{all} cores of the processor while the memory configuration 
information is assembled and passed to system software~\cite{Gottscho:2017:Measuring}. 

To measure the system impacts of the memory decoding and logging overheads, we
use the \selfish~\cite{Hoefler:2010:Characterizing} system noise measurement
microbenchmark. \selfish tracks the periods of time (\emph{detours}) when the CPU is
taken from the application to perform system tasks, by continuously reading the processor 
timestamp counter (TSC)~\cite{IntelGuide}.  When the counter interval exceeds a user-defined
threshold~\footnote{For the data presented in this section, we used 150 nanoseconds}, the time 
and duration
of this detour is recorded.

\subsection{Simulating Correctable Overheads}

In general, the communication structure of Message Passing Interface (MPI)
programs cannot be determined offline because message matches cannot be
established statically~\cite{bronevetsky2009communication}.  This makes
modeling application performance analytically challenging even if all
parameters of the application (e.g., the complete communication structure and
all relative inter-process timings) are known.  We therefore use a validated
discrete-event simulation framework to evaluate the impact of local
correctable error mitigation activities on the performance of real applications.
%for real applications via their message traces.

Our simulation-based approach models correctable error mitigation activities as
CPU detours: periods of time during which the CPU is taken from the application
and used to compute and commit checkpoint data.  This approach allows a level of
fidelity and control not always possible in implementation-based approaches. It
also allows us to examine simulated systems much larger than those generally
available.

Our simulation framework is based on the freely available
\LogGOPSim~\cite{Hoefler:2010:LogGOPSim} and the tool chain provided  by Levy et
al.~\cite{Levy2013UsingSimulation}.  \LogGOPSim uses the LogGOPS model, an
extension of the well-known LogP model~\cite{Culler:1993:LogP}, to account for
the temporal cost of communication events.  An application's communication
events are generated from traces of the application's execution.  These traces
contain the sequence of MPI operations invoked by each application process.
\LogGOPSim uses these traces to reproduce all communication dependencies,
including indirect dependencies between processes which do not communicate
directly.

\LogGOPSim can also extrapolate traces from small application runs; a trace
collected by running the application with $p$ processes can be extrapolated to
simulate performance of the application running with $k\cdot p$ processes. The
extrapolation produces exact communication patterns for MPI collective
operations and approximates point-to-point
communications~\cite{Hoefler:2010:LogGOPSim}.  The validation of \LogGOPSim and
its trace extrapolation features have been documented
previously~\cite{Hoefler:2010:LogGOPSim}, along with the simulators ability to
accurately predict application performance in the presence of performance
perturbations~\cite{Ferreira:2014:Understanding,Levy2013UsingSimulation,Hoefler:2010:Characterizing}

\begin{table}
\centering
\begin{tabular}{ l c }
\toprule
LogGOPS parameter & Cray XC40 \\
\midrule
\textcolor{red}{L}atency                & 1.8$\mu s$ \\
\textcolor{red}{o}verhead per message   & 12.4$\mu s$ \\
\textcolor{red}{g}ap per message        & 2.6$\mu s$  \\
\textcolor{red}{G}ap per byte           & 1$ns$     \\
\textcolor{red}{O}verhead per byte      & 0$ns$     \\
\textcolor{red}{S}: rendezvous threshold& 65,536 bytes \\
\bottomrule
\end{tabular}
\vspace{.6em}
\caption{
  LogGOPS parameters used in our study which roughly correspond to a Cray
  XC40 architectures.
}
\label{tab:logp}
\end{table}

\subsection{Simulation Setup and Repeatability}

To generate the data presented in this paper, we collected execution traces for
128 node runs for each applications described in \Cref{tab:app_desc} system.  We
simulated the collected native traces~\footnote{All traces from non-export
controlled application can found online, details in \Cref{sec:appendix}} with
\LogGOPSim using parameters found in \Cref{tab:logp} which roughly correspond to
a Cray XC40 architecture, a modern leadership class system. We then verified
that the simulator accurately reproduces (within 6\%) the execution time on the
respective system.

We model correctable failures using the OS noise injection functionality of
\LogGOPSim.  Using the error injection interface described previously, we
characterize the cost of correctable logging and decoding under a number
different scenarios.  Using costs measure in this work as well as costs
detailed in previous works~\cite{Gottscho:2017:Measuring} along with previously
published DRAM correctable error
rates~\cite{Li10,Hwang12,Sridharan13,Bautista-Gomez:2016:Unprotected}, we create
OS \emph{traces}~\footnote{Correctable error noise traces are also available
online, details in \Cref{sec:appendix}} that simulate the cost of correctables
while running the application trace.  Therefore, processes experiencing a
correctable error will be appropriately delayed when injected along with any
communicating processes, as detailed in \Cref{fig:propagation}.

We examine the performance of a number of HPC workloads.  These workloads,
described in \Cref{tab:app_desc}, include three important DOE production
applications (LAMMPS, CTH, and SPARC), an important HPC benchmark (HPCG), a
proxy application (LULESH) from the Department of Energy's Exascale Co-Design
Center for Materials in Extreme Environments~(ExMatEx), a scientific code used
to study the behavior of subatomic particles~(MILC), and a
mini-application~(miniFE) from Sandia's Mantevo suite.  This is a diverse set of
workloads that captures a wide range of computational methods and application
behaviors.  It additionally captures a significant cross-section of the
scalable, high-performance applications that are run on current extreme-scale
systems as well as workloads that represent the computational patterns that are
expected to be run on future systems. 

%%% removing as we are not talking about microbecnhmarks ...yet
%Lastly, we also use a number of MPI
%collective microbenchnarks to examine collective algorithm impacts.  The
%pseudocode for these microbenchmarks can be found in \Cref{alg:microbenchmark}.

\newcommand{\appDescWidth}{10.5cm}

\begin{table*}[ht!]
\centering
\begin{tabular}{@{}lc@{}}
\toprule
Application & Description \tabularnewline
\midrule
% NOTE: SNAP is not part of the current LAMMPS distribution
%LAMMPS & \parbox{\appDescWidth}{Large-scale Atomic/Molecular Massively Parallel Simulator (LAMMPS).
        LAMMPS & \parbox{\appDescWidth}{\tiny{A classical molecular dynamics simulator from Sandia National  
                                Laboratories~\cite{Plimpton:1995:Fast, LAMMPS_web}.  The data presented in                           
                                this paper are from experiments that use the Lennard-Jones (LAMMPS-lj)                               
                                potential that is included with the LAMMPS distribution.}}\\
  & \\                          
%LULESH & \parbox{\appDescWidth}{Livermore Unstructured Lagrangian Explicit Shock Hydrodynamics (LULESH).  
        LULESH & \parbox{\appDescWidth}{\tiny{A proxy application from the Department of Energy Exascale Co-Design Center
                                for Materials in Extreme Environments (ExMatEx).  LULESH approximates the
                                hydrodynamics equations discretely by partitioning the spatial problem domain
                                into a collection of volumetric elements defined by a
                                 mesh~\cite{LULESH_web}.}}\\
  & \\
        HPCG & \parbox{\appDescWidth}{\tiny{A benchmark that generates and solves a synthetic 3D sparse linear system using
                              a local symmetric Gauss-Seidel preconditioned conjugate gradient
                              method~\cite{HPCG_web}.}}\\
  & \\
        CTH & \parbox{\appDescWidth}{\tiny{A multi-material, large deformation, strong shock wave, solid mechanics
                             code~\cite{McGlaun:1990:CTH, Hertel:93:CTH} developed at Sandia National 
                             Laboratories.  The data presented in this paper are from experiments that use
                             an input that describes the simulation of the detonation of a conical explosive
                             charge (CTH-st).}}\\
  & \\
        MILC & \parbox{\appDescWidth}{\tiny{A large scale numerical simulation to study quantum chromodynamics~(QCD), the theory
                              of the strong interactions of subatomic physics~\cite{MILC_web}.}}\\
  & \\
        miniFE & \parbox{\appDescWidth}{\tiny{A proxy application that captures the key behaviors of unstructured implicit
                                finite element codes~\cite{Heroux09Mantevo}.}}\\
  & \\
        SPARC & \parbox{\appDescWidth}{\tiny{SPARC~\cite{Howard:2017:Sparc} is a next-generation compressible
        computational fluid dynamics (CFD) code being developed by Sandia National Laboratories
        as part of the NNSA's Advanced Technology Development and Mitigation (ATDM) subprogram.
        SPARC solves the Navier-Stokes and Reynolds-Averaged Navier-Stokes (RANS turbulence models)
        equations on structured and unstructured grids and is targeted towards the transonic flow
        regime to support gravity bomb analyses and the hypersonic flow regime to analyze re-entry
        vehicle analyses. In this work, the ``Generic Reentry Vehicle'' (GRV) input problem was
        used.}}\\

\bottomrule
\end{tabular}
\caption{Descriptions of the workloads used in evaluation.}
\label{tab:app_desc}
\end{table*}

%\begin{figure}[ht!]
%\centering
%%\begin{minipage}[t]{0.90\textwidth}
%\begin{algorithm}[H]
%\caption*{\textbf{Collective operation microbenchmark}}\label{alg:collectives}
%\begin{algorithmic}
%%\State{interval\_duration $\in \lbrace 50ms, 500ms, 5s, 50s\rbrace$}
%        \State{collective\_operation $\in \lbrace$ \texttt{Stencil}\\
%                                          \hspace{2.35cm}\MPIAllreduce, \MPIAlltoall, \\ 
%                                          \hspace{2.35cm}\MPIBcast, \MPIReduce$\rbrace$}
%\State
%\Procedure{collective\_micro}{collective\_operation}
%\ForAll{intervals}
%    \State \emph{execute} collective\_operation
%    \State \emph{sleep} $100ms$
%\EndFor
%\EndProcedure
%\end{algorithmic}
%\end{algorithm}
%%\end{minipage}
%        \captionof{algorithm}{Pseudocode of collective operation microbenchmark.
%        \kbf{Modify to match data}}
%\label{alg:microbenchmark}
%\end{figure}

\subsection{Simulated Correctable Error Rates}

\kbf{Add a discussion and justification here for rates we use} 

\kbf{Median might be a better value in the table below as the distribution is
skewed (errors coming from lots of "noisy" nodes), but that is more difficult
to determine}

\begin{table*} 
        \centering 
        \begin{tabular}{ l c c c c } 
         \toprule
                System & CEs / node / year & GiB / node & CEs / GiB / year & $MTBCE_{node}$ (hours)\\
         \midrule
                Schroeder et al~\cite{Schroeder:09:dram}  & 22,696 & 1--4 & 11,384 & 0.38\\
                Facebook~\cite{meza:2015:revisiting} & 5,964 & 2--24 & 460
                (median 108)& 1.47\\ % median 108 CEs/ node / year
%               Blue Waters~\cite{bluewaters} & 38.94 & 64 & 0.61 & 227.2 \\ %
                %1.04PB, 32Ki nodes
                Cielo~\cite{levy:2018:lessons} & 26.35 & 32 & 0.82 & 333 \\ %
                %385Tib, 8Ki nodes
                Trinity~\cite{Trinity} (hypothetical w/ $CE_{Cielo}$)  &  89.6 &
                128 & 0.82 & 86.5\\   % 2 PiB, 16Knodes
                Summit~\cite{Summit} (hypothetical w/ $CE_{Cielo}$)  &  425.6 &
                608 & 0.82 & 17.3\\   % 10 PiB, 4Knodes
                Exascale (hypothetical w/ $CE_{Cielo}$) & 574 & 700 & 0.82 &
                15.4 \\% 16PiB, 16Ki nodes
                Exascale (hypothetical w/ $CE_{Cielo \times 10}$) & 5,740 & 700 &
                8.2 & 1.54 \\% 16PiB, 16Ki nodes
                Exascale (hypothetical w/ $CE_{median( Facebook )}$) & 75,600 & 700 & 108 &
                0.12 \\% 16PiB, 16Ki nodes
%               Exascale (hypothetical w/ $CE_{Blue Waters}$) & 
%               427 & 700 & 0.61 & 20.41  \\% 16PiB, 16Ki nodes
         \bottomrule
        \end{tabular}
        \vspace{.6em}
        \caption{ 
                Measured and hypothesized correctable error rates
        }
        \label{tab:CE_rate}
\end{table*}




\section{Results}
\label{sec:results}

In this section we outline the costs and impacts of DRAM correctable errors
using the test infrastructure structure outlined in the previous sections.  First,
we examine the correction, logging and decoding costs of these correctable
errors.  This characterization defines the cost of each DRAM error under a number
of different configuration.  Then, using this correctable error cost
characterization and the fault rate measured in a number of published studies as
a baseline, we demonstrate the performance impacts of correctable errors on
current and future systems.

\subsection{DRAM Correctable Costs}

In this section we outline the cost of DRAM correctable on a modern HPC system.
We are concerned with three costs: the hardware error correction costs (the cost
of the ECC code to correct), logging/decoding in the OS, and logging/decoding in
firmware.

To carry out this testing, we use the \blake system located at \detail{Sandia
National Laboratories}.  \blake is 48 node Intel Skylake system connected by
\kbf{ADD network detail}.  Each node consists of 4, 24 core, 2.1GHz Intel
Skylake processors (a total of 96 cores/node) and 175GB of DDR4 DRAM per node.
This system is running Red Hat Enterprise Linux Server release 7.4 and a Linux
3.10.693 version kernel.

To inject correctable DRAM errors we use the Error INJection (EINJ) facility of
the ACPI Platform Error Interface (APEI)~\cite{ACPISpec} described previously.
We also use the \selfish operating system noise (or \emph{jitter}) measurement
microbenchmark~\footnote{See appendix \Cref{sec:appendix} for further details on
these utilities}.

\begin{figure*}
\centering{
        \subfloat[Native OS Signature for \blake]{
                \includegraphics[ width=0.22\textwidth ]
                        {blake-native-selfish-cpu0_dtr}
                \label{blake:native}
        }
        \subfloat[``Dry Run'' Injection OS Signature ]{
                \includegraphics[ width=0.22\textwidth ]
                        {blake-10sec-dry_run-selfish-cpu0_dtr}
                \label{blake:dry-run}
        }
        \subfloat[Software Cost (OS decoding with CMCI)]{
                \includegraphics[ width=0.22\textwidth ]
                        {blake-10sec-edac-ignore_ce-log-correctable-selfish-cpu0_dtr}
                \label{blake:OS_log}
        }
        \subfloat[Firmware Cost (Firmware decoding with EMCA, threshold set to 10)]{
                \includegraphics[ width=0.22\textwidth ]
                        {blake-10sec-extlog-ignore_ce-selfish-cpu0_dtr}
                \label{blake:FW_log}
        }
}
\caption{
        Native and ``dry run'' OS noise signature for \blake.  The ''dry run''
        option configures the EINJ interface at the requested frequency (in
        this case every ten seconds) but does not trigger the error.  This
        attempts to measure the cost of the error injection utility and writing
        to the {\texttt{sysfs}} filesystem.  As can be seen from the figure,
        the injection utility impacts no additional OS noise.  \kbf{Add
        decription of injection plots.  missing is plot with ``All logging
        turned off''.  This plot looks the same as native signature}
}
\label{fig:baselines}
\end{figure*}

As we are interested in measuring the noise signature of DRAM correctables, we
first must measure the native noise signature of the system, as well as
ensuring the error injection utility does not impart additional OS noise costs
that would not be part of an actual correctable DRAM error.
\Cref{fig:baselines} shows the native OS noise signature on the \blake system
and a ``dry-run'' execution of our error injection utility.  The dry-run option
of the injection utility configures the injection interface to triggers errors
at the requested interval, in this case we arbitrarily chose every 10 seconds,
but does not trigger the error.  An OS noise event detected by \selfish is
denoted by a spike in the figure, the X-Axis being the time the noise event
occurred and the duration of the noise event being the amplitude on the Y-Axis.
From this figure we can see that the native noise and dry-run signature are
nearly identical.  Therefore the injection framework does not impart any
additional OS noise over the systems native signature.  \kbf{Add details for
\Cref{blake:OS_log,blake:FW_log}.  Hardware loggin plot is not shown (with all
logging turned off).  this plot is the same as the "native" signature}

\subsection{Exploring Correctable Costs}  

\kbf{Add plots here ... and an explanation}

\begin{figure*}
\centering{
    \subfloat[Correctables at current rate, 16KNodes]{
        \includegraphics[width=0.3\textwidth]{1.78571e-06_16384-apps-delta.pdf}
        \label{fig:apps:current}
    }
    \subfloat[Correctables at {\bf{10}}X the current rate, 16KNodes]{
        \includegraphics[width=0.3\textwidth]{1.78571e-05_16384-apps-delta.pdf}
        \label{fig:apps:10xcurrent}
    }
    \subfloat[Correctables at {\bf{100}}X the current rate, 16KNodes]{
        \includegraphics[width=0.3\textwidth]{0.000178571_16384-apps-delta.pdf}
        \label{fig:apps:100Xcurrent}
    } \\
    \subfloat[Correctables at {\bf{1000}}X the current rate, 16KNodes]{
        \includegraphics[width=0.3\textwidth]{0.00178571_16384-apps-delta.pdf}
        \label{fig:apps:1000Xcurrent}
    }
    \subfloat[Correctables at {\bf{10,000}}X the current rate, 16KNodes]{
        \includegraphics[width=0.3\textwidth]{0.0178571_16384-apps-delta.pdf}
        \label{fig:apps:10000Xcurrent}
    }
    \subfloat[Correctables at {\bf{100,000}}X the current rate, 16KNodes]{
        \includegraphics[width=0.3\textwidth]{0.178571_16384-apps-delta.pdf}
        \label{fig:apps:100000Xcurrent}
    }
}
\caption
{
        \kbf{Add details}
}
\label{fig:apps-delta:16K}
\end{figure*}

\begin{figure*}
\centering{
    \subfloat[Correctables at current rate, 16KNodes]{
        \includegraphics[width=0.3\textwidth]{1.78571e-06_16384-micros-delta.pdf}
        \label{fig:micros:current}
    }
    \subfloat[Correctables at {\bf{10}}X the current rate, 16KNodes]{
        \includegraphics[width=0.3\textwidth]{1.78571e-05_16384-micros-delta.pdf}
        \label{fig:micros:10xcurrent}
    }
    \subfloat[Correctables at {\bf{100}}X the current rate, 16KNodes]{
        \includegraphics[width=0.3\textwidth]{0.000178571_16384-micros-delta.pdf}
        \label{fig:micros:100Xcurrent}
    } \\
    \subfloat[Correctables at {\bf{1000}}X the current rate, 16KNodes]{
        \includegraphics[width=0.3\textwidth]{0.00178571_16384-micros-delta.pdf}
        \label{fig:micros:1000Xcurrent}
    }
    \subfloat[Correctables at {\bf{10,000}}X the current rate, 16KNodes]{
        \includegraphics[width=0.3\textwidth]{0.0178571_16384-micros-delta.pdf}
        \label{fig:micros:10000Xcurrent}
    }
    \subfloat[Correctables at {\bf{100,000}}X the current rate, 16KNodes]{
        \includegraphics[width=0.3\textwidth]{0.178571_16384-micros-delta.pdf}
        \label{fig:micros:100000Xcurrent}
    }
}
\caption
{
        \kbf{Add details}
}
\label{fig:micros-delta:16K}
\end{figure*}

\subsection{Impact of Scale}

\kbf{Trim down all these figures ... and discuss results.  Also check which
applications are here.  Prolly remove GTC and add SPARC}

\begin{figure*}
\centering{
    \subfloat[Hardware Correction Impacts]{
        \includegraphics[width=0.3\textwidth]{allHz_1.5e-7secs}
        \label{fig:current:hardware}
    }
    \subfloat[Software-based Logging Impacts]{
        \includegraphics[width=0.3\textwidth]{1.7857e-06Hz_0.00077secs}
        \label{fig:current:cmca}
    }
    \subfloat[Firmware-based Logging Impacts]{
        \includegraphics[width=0.3\textwidth]{1.7857e-06Hz_0.13secs}
        \label{fig:current:firmware}
    }}
   \caption{\textbf{Performance impacts of correctable errors using the current
                    correctable error rate from \cielo.  Three scenarios are shown:
                    hardware only correction with no logging ($150ns$ per event),
                    Software-based logging using the Corrected Machine Check
                    Architecture (CMCA) ($775\mu sec$ per event), and the
                    Firmware-based logging
                    using the Enhanced Machine Check Architecture (EMCA)
                    ($133msecs$ per event) }.
   }
\label{fig:current}
\end{figure*}

\begin{figure*}
\centering{
    \subfloat[Hardware Correction Impacts]{
        \includegraphics[width=0.3\textwidth]{allHz_1.5e-7secs}
        \label{fig:10X:hardware}
    }
    \subfloat[Software-based Logging Impacts]{
        \includegraphics[width=0.3\textwidth]{1.7857e-05Hz_0.00077secs}
        \label{fig:10X:cmca}
    }
    \subfloat[Firmware-based Logging Impacts]{
        \includegraphics[width=0.3\textwidth]{1.7857e-05Hz_0.13secs}
        \label{fig:10X:firmware}
    }}
   \caption{\textbf{Performance impacts of correctable errors using $10\times$ the current
                    correctable error rate from \cielo.  Three scenarios are shown:
                    hardware only correction with no logging ($150ns$ per event),
                    Software-based logging using the Corrected Machine Check
                    Architecture (CMCA) ($775\mu sec$ per event), and the
                    Firmware-based logging
                    using the Enhanced Machine Check Architecture (EMCA)
                    ($133msecs$ per event) }.
   }
\label{fig:10X}
\end{figure*}

\begin{figure*}
\centering{
    \subfloat[Hardware Correction Impacts]{
        \includegraphics[width=0.3\textwidth]{allHz_1.5e-7secs}
        \label{fig:100X:hardware}
    }
    \subfloat[Software-based Logging Impacts]{
        \includegraphics[width=0.3\textwidth]{0.00017857Hz_0.00077secs}
        \label{fig:100X:cmca}
    }
    \subfloat[Firmware-based Logging Impacts]{
        \includegraphics[width=0.3\textwidth]{0.00017857Hz_0.13secs}
        \label{fig:100X:firmware}
    }}
   \caption{\textbf{Performance impacts of correctable errors using $100\times$ the current
                    correctable error rate from \cielo.  Three scenarios are shown:
                    hardware only correction with no logging ($150ns$ per event),
                    Software-based logging using the Corrected Machine Check
                    Architecture (CMCA) ($775\mu sec$ per event), and the
                    Firmware-based logging
                    using the Enhanced Machine Check Architecture (EMCA)
                    ($133msecs$ per event) }.
   }
\label{fig:100X}
\end{figure*}

\begin{figure*}
\centering{
    \subfloat[Hardware Correction Impacts]{
        \includegraphics[width=0.3\textwidth]{allHz_1.5e-7secs}
        \label{fig:1000X:hardware}
    }
    \subfloat[Software-based Logging Impacts]{
        \includegraphics[width=0.3\textwidth]{0.0017857Hz_0.00077secs}
        \label{fig:1000X:cmca}
    }
    \subfloat[Firmware-based Logging Impacts]{
        \includegraphics[width=0.3\textwidth]{0.0017857Hz_0.13secs}
        \label{fig:1000X:firmware}
    }}
   \caption{\textbf{Performance impacts of correctable errors using $1000\times$ the current
                    correctable error rate from \cielo.  Three scenarios are shown:
                    hardware only correction with no logging ($150ns$ per event),
                    Software-based logging using the Corrected Machine Check
                    Architecture (CMCA) ($775\mu sec$ per event), and the
                    Firmware-based logging
                    using the Enhanced Machine Check Architecture (EMCA)
                    ($133msecs$ per event) }.
   }
\label{fig:1000X}
\end{figure*}

\subsection{One Bursty Node}
\begin{figure*}
\centering{
    \subfloat[Hardware Correction Impacts]{
            \includegraphics[width=0.3\textwidth]{{5Hz_1.5e-07sec-apps-scale}}
        \label{fig:hardware}
    }
    \subfloat[Software-based Logging Impacts]{
            \includegraphics[width=0.3\textwidth]{{5Hz_0.000755sec-apps-scale}}        \label{fig:cmca}
    }
    \subfloat[Firmware-based Logging Impacts]{
            \includegraphics[width=0.3\textwidth]{{5Hz_0.133sec-apps-scale}}
        \label{fig:firmware}
    }}
       \caption{\textbf{Performance impacts of one 5HZ bursty correctable error
                    node. Three scenarios are shown:
                    hardware only correction with no logging ($150ns$ per event),
                    Software-based logging using the Corrected Machine Check
                    Architecture (CMCA) ($775\mu sec$ per event), and the
                    Firmware-based logging
                    using the Enhanced Machine Check Architecture (EMCA)
                    ($133msecs$ per event) }.
   }
\end{figure*}

\begin{figure*}
\centering{
        \includegraphics[width=0.8\textwidth]{{0.000775sec_16384nodes-apps-freq}}
    }
   \label{fig:775us}
   \caption{\textbf{Performance impacts of  bursty correctable errors for
        Software-based logging using the Corrected Machine Check
                    Architecture (CMCA) ($775\mu sec$ per event) for a
                    number of frequencies from 5Hz to 50KHz}
   }
\end{figure*}

\begin{figure*}
\centering{
        \includegraphics[width=0.8\textwidth]{{5Hz_16384nodes-apps-delta}}
    }
   \label{fig:5Hz:delta}
   \caption{\textbf{Performance impacts of bursty correctable errors at
        5Hz as a function of the cost per event.}
   }
\end{figure*}

\subsection{Exploring Correctable Frequency and Duration}

\kbf{Add plots here ... }

\subsection{Discussion and Analysis}

\kbf{Add the applciation analysis stuff here}


%\input{lessons}


\section{Related Work}
\label{sec:related}


\kbf{This first paragraph is nearly identical to what is in the parallel
submission}

Efforts to characterize the frequency and type of correctable and uncorrectable
failures on large-scale HPC and cloud systems have been ongoing for over a
decade now.  Schroeder and Gibson studied failures in supercomputer systems at
LANL~\cite{Schroeder:2006:Large-scale}.  Schroeder {\it et al.}~conducted a
large-scale field study using Google's server fleet~\cite{Schroeder09}.  Li {\it
et al.}~studied memory errors on three different data sets, including a server
farm of an Internet service provider~\cite{Li07}.  In 2010, Li {\it et
al.}~published an expanded study of memory errors at an Internet server farm and
other data center sources~\cite{Li10}.  Hwang {\it et al.}~published an expanded
study on Google's server fleet, as well as two IBM Blue Gene
clusters~\cite{Hwang12}.  Sridharan and Liberty presented a study of DRAM
failures in a high-performance computing system~\cite{Sridharan12}.  El-Sayed
{\it et al.}~studied temperature effects of DRAM in data center
environments~\cite{Elsayed12}.  Similarly, Siddiqua {\it et al.}~studied DRAM
failures from client and server systems~\cite{Siddiqua13}.  Sridharan {\it et
al.}~studied DRAM and SRAM faults, with a focus on positional and vendor
effects~\cite{Sridharan13}.  Di Martino {\it et al.}~studied failures in Blue
Waters, an HPC system at the University of Illinois,
Urbana-Champaign~\cite{bluewaters}.  Additionally, Bautista-Gomez {\it et
al.}~\cite{Bautista-Gomez:2016:Unprotected} presented a study of DRAM memory
errors on a large-scale system in the explicit absence of an ECC in an effort
understand the behavior of raw memory failures. \kbf{ADD Christian's SC17 paper}.
Finally, \kbf{ADD our SC18 paper and DFT '17 paper}

Our study has origins in previous works that characterizes application behavior
in the presence of OS
noise~\cite{Ferreira:2008:Characterizing,Hoefler:2010:Characterizing}.
Collectively, this research shows that the pattern of OS noise events determines
the impact on application performance and the benefits of coordination.
Moreover, it shows that the duration of an OS noise event can significantly
slowdown application performance.  Additionally, Delgado and
Karavanic~\cite{Delgado:2013:SMM} examined the impact of system management mode
(SMM) interrupts on network and IO workloads.  Similarly, Macarenco {\it et
al.}~\cite{Macarenco:2016:Effects} examine the impact of SMM mode for small
scale NAS parallel benchmark runs and the UnixBench benchmark.


Most closely related, Gottscho {\it et al.}~\cite{Gottscho:2017:Measuring}
examine the single-machine performance impacts of correctable DRAM errors for
web search and SPEC CPU2006 benchmarks.  In this this work the authors
demonstrate that an "avalanche" of these correctable errors can significantly
impact benchmark performance.  Finally, using a proprietary hardware and
software tool, the authors also characterize the impacts of DRAM correctable
errors for Windows Server 2012 at a hardware, firmware, and OS level.


\kbf{Strengthen this paragraph on why we are novel and awesome}

Our work distinguishes itself from these existing studies in server al important
ways.  First, to the best of our knowledge this is the first study to examine
the HPC performance impacts of \emph{correctable} DRAM errors.  Much of the existing
work is focused on either characterizing failures or examining the application
performance implication of DUEs. In addition, unlike previous work, we attempt a
principled analysis of the correctable error rate increases likely with future
systems as well as the importance of reducing the duration of each memory error
event.  Finally, we examine how an applications communication dependencies,
particularly its collective communication, influence the impacts from correctable
errors \kbf{ADD brief application analysis?}.

\kbf{Anything more to add here?}


%\input{conclusion}


% For peer review papers, you can put extra information on the cover
% page as needed:
% \ifCLASSOPTIONpeerreview
% \begin{center} \bfseries EDICS Category: 3-BBND \end{center}
% \fi
%
% For peerreview papers, this IEEEtran command inserts a page break and
% creates the second title. It will be ignored for other modes.
%\IEEEpeerreviewmaketitle

% use section* for acknowledgment
\opt{NotBlind}{
\section*{Acknowledgment}
\kbf{Add blurbs}
}

\bibliographystyle{IEEEtran}
\bibliography{all} 

\end{document}
