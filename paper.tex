% R&A Tracking number: 
\documentclass[sigconf,review]{acmart}

\usepackage{url}
\usepackage{amsmath}
\usepackage{caption}
\usepackage[caption=false,font=footnotesize]{subfig}
\usepackage{graphicx}
\usepackage{verbatim}
\usepackage{xcolor}
\usepackage{xspace}
\usepackage{booktabs}
\usepackage{multirow}
\usepackage{grffile}
\usepackage{tikz}
\usepackage{cleveref}
\usepackage[Blind]{optional}
% Prevent single column figures from being rendered before double column figures
\usepackage{fixltx2e}
\usepackage{algorithm}
\usepackage{algpseudocode}

\usepackage{fancyvrb}

\graphicspath{{figs/}}

\opt{Blind}{
        \newcommand{\blake}{{\bf{[development system name]}}\xspace}
        \newcommand{\cielo}{{\bf{[production system name]}}\xspace}
        \newcommand{\detail}[1]{{\bf{[detail removed for double-blind review]}\xspace}}
}

\opt{NotBlind}{
        \newcommand{\blake}{Blake\xspace}
        \newcommand{\cielo}{Cielo\xspace}
        \newcommand{\detail}[1]{#1}
}

\newcommand{\sll}[1]{\textcolor{blue}{[sll: #1]}}
\newcommand{\kbf}[1]{\textcolor{red}{[kbf: #1]}}

%
% Some common definitions
%

\providecommand{\email}[1]{\url{#1}}
\newcommand{\LogGOPSim}{{\texttt{Log\-GOP\-Sim}}\xspace}

%
% ------------------------------------------------------------------------------
% Help functions for references
%
\providecommand{\assignmentname}{Exercise}
\providecommand{\equationname}{Equation}
\providecommand{\figurename}{Figure}
\providecommand{\figuresname}{Figures}
\providecommand{\tablename}{Table}
\providecommand{\nameofsection}{Section}
\newcommand{\reffig}[1]{\figurename~\ref{#1}}
\newcommand{\refsubfig}[1]{\figurename~\subref{#1}}
\newcommand{\reftwosubfig}[2]{\figuresname~\subref{#1} and \subref{#2}}
\newcommand{\reftwofig}[2]{\figuresname~\ref{#1} and \ref{#2}}
\newcommand{\refthreesubfig}[3]{\figuresname~\subref{#1}, \subref{#2}, and \subref{#3}}
\newcommand{\refthreefig}[3]{\figuresname~\ref{#1}, \ref{#2}, and \ref{#3}}
\newcommand{\reftab}[1]{\tablename~\ref{#1}}
\newcommand{\reflst}[1]{\lstlistingname~\ref{#1}}
\newcommand{\refchap}[1]{\chaptername~\ref{#1}}
\newcommand{\refapp}[1]{\appendixname~\ref{#1}}
%\newcommand{\refsec}[1]{\sectionname~\ref{#1}}
\newcommand{\refsec}[1]{\nameofsection~\ref{#1}}
\newcommand{\refeq}[1]{\equationname~\ref{#1}}
\newcommand{\refpage}[1]{on page~\pageref{#1}}


\newlength{\fullfigWidth}
\setlength{\fullfigWidth}{0.855\columnwidth}

\newlength{\twoacross}
\setlength{\twoacross}{0.4\textwidth}

\newlength{\threeacross}
\setlength{\threeacross}{0.315\textwidth}

%
% ------------------------------------------------------------------------------
% Help functions for editing
%
\definecolor{NoteColor}{rgb}{0.71, 0.22, 0.21}
\newcommand{\note}[1]{{\color{NoteColor}\bf Note: }{\color{NoteColor} #1}}
\newcommand{\hl}[1]{\textcolor{red}{\texttt{#1}}}



%
% ------------------------------------------------------------------------------
% General
%
\newcommand*{\OS}{operating system\xspace}
\newcommand*{\OSes}{operating systems\xspace}
\newcommand{\rmpi}{\textit{r}MPI\xspace}



%
% Programming
%
\newcommand*{\file}[1]{{\small\texttt{#1}}\xspace}
\newcommand*{\cmd}[1]{{\small\texttt{#1}}\xspace}
\newcommand*{\var}[1]{{\small\texttt{#1}}\xspace}
\newcommand*{\keyword}[1]{{\small\texttt{#1}}\xspace}
% \newcommand*{\func}[1]{\texttt{\textcolor{funcfg}{#1()}}\xspace}
\newcommand*{\func}[1]{{\small\texttt{#1()}}\xspace}
% \newcommand*{\mfunc}[1]{\texttt{\textcolor{funcfg}{#1()}}\index{#1 (func)}\index{MPI functions!#1}\xspace}
\newcommand*{\mfunc}[1]{{\small\texttt{#1()}}\index{#1 (func)}\index{MPI functions!#1}\xspace}
\newcommand*{\pfunc}[1]{{\small\texttt{#1()}}\index{#1 (func)}\index{Pthreads functions!#1}\xspace}
\newcommand*{\code}[1]{{\small\texttt{#1}}\xspace}
\renewcommand*{\arg}[1]{{\small\texttt{#1}}\xspace}
\newcommand*{\type}[1]{{\small\texttt{#1}}\index{#1 (type)}\index{data types!#1}\xspace}
\newcommand*{\mtype}[1]{{\small\texttt{#1}}\index{#1 (type)}\index{MPI data types!#1}\xspace}
\newcommand*{\ptype}[1]{{\small\texttt{#1}}\index{#1 (type)}\index{pthreads data types!#1}\xspace}
\newcommand*{\mconst}[1]{{\small\texttt{#1}}\index{#1 (const.)}\index{MPI constants!#1}\xspace}
\newcommand*{\pconst}[1]{{\small\texttt{#1}}\index{#1 (const.)}\index{pthreads constants!#1}\xspace}

\newcommand*{\MAXINT}{\type{MAX\_INT}\xspace}
\newcommand*{\NULL}{\type{NULL}\xspace}




%
% MPI types
%
\newcommand*{\MPIStatus}{\mtype{MPI\_\-Status}}
\newcommand*{\MPISOURCE}{\mtype{MPI\_\-SOURCE}}
\newcommand*{\MPITAG}{\mtype{MPI\_\-TAG}}



%
% MPI data types
%
\newcommand*{\MPICHAR}{\mtype{MPI\_\-CHAR}}
\newcommand*{\MPISIGNEDCHAR}{\mtype{MPI\_\-SIGNED\_\-CHAR}}
\newcommand*{\MPISHORT}{\mtype{MPI\_\-SHORT}}
\newcommand*{\MPIINT}{\mtype{MPI\_\-INT}}
\newcommand*{\MPIINTEGER}{\mtype{MPI\_\-INTEGER}} % Fortran only
\newcommand*{\MPILONG}{\mtype{MPI\_\-LONG}}
\newcommand*{\MPILONGLONGINT}{\mtype{MPI\_\-LONG\_\-LONG\_\-INT}}
\newcommand*{\MPILONGLONG}{\mtype{MPI\_\-LONG\_\-LONG}} % Exists due to an error in MPI-2.0. Should be MPI_LONG_LONG_INT
\newcommand*{\MPIUNSIGNEDLONGLONG}{\mtype{MPI\_\-UNSIGNED\_\-LONG\_\-LONG}}
\newcommand*{\MPIUNSIGNEDCHAR}{\mtype{MPI\_\-UNSIGNED\_\-CHAR}}
\newcommand*{\MPIUNSIGNEDSHORT}{\mtype{MPI\_\-UNSIGNED\_\-SHORT}}
\newcommand*{\MPIUNSIGNED}{\mtype{MPI\_\-UNSIGNED}}
\newcommand*{\MPIUNSIGNEDLONG}{\mtype{MPI\_\-UNSIGNED\_\-LONG}}
\newcommand*{\MPIFLOAT}{\mtype{MPI\_\-FLOAT}}
\newcommand*{\MPIDOUBLE}{\mtype{MPI\_\-DOUBLE}}
\newcommand*{\MPILONGDOUBLE}{\mtype{MPI\_\-LONG\_\-DOUBLE}}
\newcommand*{\MPIBYTE}{\mtype{MPI\_\-BYTE}}
\newcommand*{\MPIPACKED}{\mtype{MPI\_\-PACKED}}
\newcommand*{\MPIWCHAR}{\mtype{MPI\_\-WCHAR}}



%
% MPI Constants
%
\newcommand*{\MPIINPLACE}{\mconst{MPI\_\-IN\_\-PLACE}}
\newcommand*{\MPICOMMWORLD}{\mconst{MPI\_\-COMM\_\-WORLD}}
\newcommand*{\MPIANYSOURCE}{\mconst{MPI\_\-ANY\_\-SOURCE}}
\newcommand*{\MPIANYTAG}{\mconst{MPI\_\-ANY\_\-TAG}}
\newcommand*{\MPITAGUB}{\mconst{MPI\_\-TAG\_\-UB}}
\newcommand*{\MPISUCCESS}{\mconst{MPI\_\-SUCCESS}}
\newcommand*{\MPISTATUSIGNORE}{\mconst{MPI\_\-STATUS\_\-IGNORE}}
\newcommand*{\MPISTATUSESIGNORE}{\mconst{MPI\_\-STATUSES\_\-IGNORE}}
\newcommand*{\MPIWTIMEISGLOBAL}{\mconst{MPI\_\-WTIME\_\-IS\_\-GLOBAL}}
\newcommand*{\MPIUNDEFINED}{\mconst{MPI\_\-UNDEFINED}}
\newcommand*{\MPIREQUESTNULL}{\mconst{MPI\_\-REQUEST\_\-NULL}}



%
% MPI Collective Functions
%
\newcommand*{\MPIBarrier}{\mfunc{MPI\_\-Barrier}}
\newcommand*{\MPIBcast}{\mfunc{MPI\_\-Bcast}}
\newcommand*{\MPIReduce}{\mfunc{MPI\_\-Reduce}}
\newcommand*{\MPIGather}{\mfunc{MPI\_\-Gather}}
\newcommand*{\MPIScatter}{\mfunc{MPI\_\-Scatter}}
\newcommand*{\MPIAllgather}{\mfunc{MPI\_\-Allgather}}
\newcommand*{\MPIAllreduce}{\mfunc{MPI\_\-Allreduce}}
\newcommand*{\MPIAlltoall}{\mfunc{MPI\_\-Alltoall}}
\newcommand*{\MPIScan}{\mfunc{MPI\_\-Scan}}
\newcommand*{\MPIExscan}{\mfunc{MPI\_\-Exscan}}
\newcommand*{\MPIGatherv}{\mfunc{MPI\_\-Gatherv}}
\newcommand*{\MPIScatterv}{\mfunc{MPI\_\-Scatterv}}
\newcommand*{\MPIAllgatherv}{\mfunc{MPI\_\-Allgatherv}}
\newcommand*{\MPIAlltoallv}{\mfunc{MPI\_\-Alltoallv}}
\newcommand*{\MPIReducescatter}{\mfunc{MPI\_\-Reduce\_\-scatter}}
\newcommand*{\MPIAlltoallw}{\mfunc{MPI\_\-Alltoallw}}
\newcommand*{\MPIOpcreate}{\mfunc{MPI\_\-Op\_\-create}}
\newcommand*{\MPIOpfree}{\mfunc{MPI\_\-Op\_\-free}}


%
% MPI Functions
%
\newcommand*{\MPIInit}{\mfunc{MPI\_\-Init}}
\newcommand*{\MPIInitalized}{\mfunc{MPI\_\-Initialized}}
\newcommand*{\MPIFinalize}{\mfunc{MPI\_\-Finalize}}
\newcommand*{\MPICommrank}{\mfunc{MPI\_\-Comm\_\-rank}}
\newcommand*{\MPICommsize}{\mfunc{MPI\_\-Comm\_\-size}}
\newcommand*{\MPICommdup}{\mfunc{MPI\_\-Comm\_\-dup}}
\newcommand*{\MPICommsplit}{\mfunc{MPI\_\-Comm\_\-split}}
\newcommand*{\MPIRequestfree}{\mfunc{MPI\_\-Request\_\-free}}
\newcommand*{\MPIWtime}{\mfunc{MPI\_\-Wtime}}
\newcommand*{\MPISend}{\mfunc{MPI\_\-Send}}
\newcommand*{\MPIIsend}{\mfunc{MPI\_\-Isend}}
\newcommand*{\MPIRecv}{\mfunc{MPI\_\-Recv}}
\newcommand*{\MPIIrecv}{\mfunc{MPI\_\-Irecv}}
\newcommand*{\MPIWait}{\mfunc{MPI\_\-Wait}}
\newcommand*{\MPIWaitall}{\mfunc{MPI\_\-Wait\-all}}
\newcommand*{\MPIWaitany}{\mfunc{MPI\_\-Wait\-any}}
\newcommand*{\MPITest}{\mfunc{MPI\_\-Test}}
\newcommand*{\MPITestall}{\mfunc{MPI\_\-Test\-all}}
\newcommand*{\MPITestany}{\mfunc{MPI\_\-Test\-any}}
\newcommand*{\MPIProbe}{\mfunc{MPI\_\-Probe}}
\newcommand*{\MPIIprobe}{\mfunc{MPI\_\-Iprobe}}
\newcommand*{\MPIAbort}{\mfunc{MPI\_\-Abort}}
\newcommand*{\MPICancel}{\mfunc{MPI\_\-Cancel}}
\newcommand*{\MPICartcreate}{\mfunc{MPI\_\-Cart\_\-create}}
\newcommand*{\MPIStart}{\mfunc{MPI\_\-Start}}
\newcommand*{\MPIRecvinit}{\mfunc{MPI\_\-Recv\_\-init}}
\newcommand*{\MPIStartall}{\mfunc{MPI\_\-Startall}}
\newcommand*{\MPIGrouprank}{\mfunc{MPI\_\-Group\_\-rank}}
\newcommand*{\MPIGroupstar}{\mfunc{MPI\_\-Group\_$\ast$}}



%
% MPI Data Type Functions
%
\newcommand*{\MPITypecontiguous}{\mfunc{MPI\_\-Type\_\-contiguous}}
\newcommand*{\MPITypecommit}{\mfunc{MPI\_\-Type\_\-commit}}
\newcommand*{\MPITypefree}{\mfunc{MPI\_\-Type\_\-free}}
\newcommand*{\MPITypevector}{\mfunc{MPI\_\-Type\_\-vector}}
\newcommand*{\MPITypeindexed}{\mfunc{MPI\_\-Type\_\-indexed}}
\newcommand*{\MPITypegetextent}{\mfunc{MPI\_\-Type\_\-get\_\-extent}}
\newcommand*{\MPITypeextent}{\mfunc{MPI\_\-Type\_\-extent}} % MPI-1 depreciated
\newcommand*{\MPITypelb}{\mfunc{MPI\_\-Type\_\-lb}} % MPI-1 depreciated
\newcommand*{\MPITypeub}{\mfunc{MPI\_\-Type\_\-ub}} % MPI-1 depreciated
\newcommand*{\MPIGetcount}{\mfunc{MPI\_\-Get\_\-count}}
\newcommand*{\MPITypegettrueextent}{\mfunc{MPI\_\-Type\_\-get\_\-true\_\-extent}}
\newcommand*{\MPITypesize}{\mfunc{MPI\_\-Type\_\-size}}
\newcommand*{\MPITypecreateresized}{\mfunc{MPI\_\-Type\_\-create\_\-resized}}
\newcommand*{\MPITypecreatestruct}{\mfunc{MPI\_\-Type\_\-create\_\-struct}}
\newcommand*{\MPIGetaddress}{\mfunc{MPI\_\-Get\_\-address}}




% MPI ops
\newcommand*{\MPIMAX}{\code{MPI\_\-MAX\index{MPI\_\-MAX}\index{MPI predefined functions!MPI\_\-MAX}}}
\newcommand*{\MPIMIN}{\code{MPI\_\-MIN\index{MPI\_\-MIN}\index{MPI predefined functions!MPI\_\-MIN}}}
\newcommand*{\MPISUM}{\code{MPI\_\-SUM\index{MPI\_\-SUM}\index{MPI predefined functions!MPI\_\-SUM}}}
\newcommand*{\MPIPROD}{\code{MPI\_\-PROD\index{MPI\_\-PROD}\index{MPI predefined functions!MPI\_\-PROD}}}
\newcommand*{\MPILAND}{\code{MPI\_\-LAND\index{MPI\_\-LAND}\index{MPI predefined functions!MPI\_\-LAND}}}
\newcommand*{\MPIBAND}{\code{MPI\_\-BAND\index{MPI\_\-BAND}\index{MPI predefined functions!MPI\_\-BAND}}}
\newcommand*{\MPILOR}{\code{MPI\_\-LOR\index{MPI\_\-LOR}\index{MPI predefined functions!MPI\_\-LOR}}}
\newcommand*{\MPIBOR}{\code{MPI\_\-BOR\index{MPI\_\-BOR}\index{MPI predefined functions!MPI\_\-BOR}}}
\newcommand*{\MPILXOR}{\code{MPI\_\-LXOR\index{MPI\_\-LXOR}\index{MPI predefined functions!MPI\_\-LXOR}}}
\newcommand*{\MPIBXOR}{\code{MPI\_\-BXOR\index{MPI\_\-BXOR}\index{MPI predefined functions!MPI\_\-BXOR}}}
\newcommand*{\MPIMAXLOC}{\code{MPI\_\-MAXLOC\index{MPI\_\-MAXLOC}\index{MPI predefined functions!MPI\_\-MAXLOC}}}
\newcommand*{\MPIMINLOC}{\code{MPI\_\-MINLOC\index{MPI\_\-MINLOC}\index{MPI predefined functions!MPI\_\-MINLOC}}}

\newcommand*{\MPIFLOATINT}{\code{MPI\_\-FLOAT\_INT\index{MPI\_\-FLOAT\_\-INT}\index{MPI predefined datatypes!MPI\_\-FLOAT\_\-INT}}}
\newcommand*{\MPIDOUBLEINT}{\code{MPI\_\-DOUBLE\_\-INT\index{MPI\_\-DOUBLE\_\-INT}\index{MPI predefined datatypes!MPI\_\-DOUBLE\_\-INT}}}
\newcommand*{\MPILONGINT}{\code{MPI\_\-LONG\_\-INT\index{MPI\_\-LONG\_\-INT}\index{MPI predefined datatypes!MPI\_\-LONG\_\-INT}}}
\newcommand*{\MPITWOINT}{\code{MPI\_\-2INT\index{MPI\_\-2INT}\index{MPI predefined datatypes!MPI\_\-2INT}}}
\newcommand*{\MPISHORTINT}{\code{MPI\_\-SHORT\_\-INT\index{MPI\_\-SHORT\_\-INT}\index{MPI predefined datatypes!MPI\_\-SHORT\_\-INT}}}
\newcommand*{\MPILONGDOUBLEINT}{\code{MPI\_\-LONG\_\-DOUBLE\_\-INT\index{MPI\_\-LONG\_\-DOUBLE\_\-INT}\index{MPI predefined datatypes!MPI\_\-LONG\_\-DOUBLE\_\-INT}}}

\hyphenation{op-tical net-works semi-conduc-tor}

\acmConference[SC'19]{The International Conference for High Performance
Computing, Networking, Storage and Analysis}{November 17--22}{Denver, CO}

\setcopyright{none}

\begin{document}
\title{Understanding the Effects of Correctable Errors at Scale}

\opt{NotBlind}{
\author{\IEEEauthorblockN{
Kurt B. Ferreira\IEEEauthorrefmark{1},
Scott Levy\IEEEauthorrefmark{1},
Nathan DeBardeleben\IEEEauthorrefmark{2},
Sean Blanchard\IEEEauthorrefmark{2},
Victor Kuhns\IEEEauthorrefmark{1},
}
\IEEEauthorblockA{\IEEEauthorrefmark{1}Center for Computing Research, Sandia National Laboratories\\
\email{{sllevy, kbferre,vgkuhns}@sandia.gov}}
\IEEEauthorblockA{\IEEEauthorrefmark{2}
Ultrascale Systems Research Center, Los Alamos National Laboratory\\
\email{{ndebard,sblanchard}@lanl.gov}}
} % end AUTHOR block
}

\opt{Blind}{
        \author{}
}

\maketitle

\makeatletter\let\myTitle\@title\makeatother


\begin{abstract}

% Motivation
        Fault-tolerance poses a major challenge for future large-scale systems.
        Active research in the field typically focuses on mitigating the
        effects of uncorrectable errors, those fatal errors that typically
        require an application to restart from a known good state.
% Methods/Procedures
        However, the impacts of the most common error of large-scale systems,
        correctable errors, is typically overlooked.  Moreover, increased
        memory volumes and expected technology changes on future extreme-scale
        system may make these errors even more likely and more of a concern.
        In this work, we use a simulation-based approach to show how local
        correctable errors can significantly affect the performance of key
        extreme-scale workloads.
% Findings
        Our study shows that local delays due to correctable errors can
        propagate through MPI message synchronization to other processes,
        causing a cascading series of delays.  We also find that though much of
        the focus on correctable errors is focused on reducing failure rates,
        reducing the rate of each individual error may be more impacting on
        overheads at scale. Finally,  this study outlines the errors
        frequencies, durations, and scales in which performance is
        significantly impacted for a number of key extreme-scale workloads.
% Conclusion/Impacts
        This work provides critical analysis and insight into the overheads of
        common correctable errors and provides practical advice to users and
        systems administrators in an effort to fine-tune performance to
        application and system characteristics.  \end{abstract}




\section{Introduction}
\label{sec:intro}

\kbf{Some of these paragraphs may need to be shuffled a bit, order may not be the
most logical}

\kbf{Following paragraph is nearly identical to parallel submission}
\sll{First paragraph is very close to SC18 paper.}

Maintaining the performance of high-performance computing~(HPC) applications as
failures become more and more frequent is a major challenge that needs to be
addressed for next-generation extreme-scale systems.  Many recent studies have
demonstrated that hardware failures are expected to become ever more
common~\cite{Bergman08exascalecomputing}.  Increasing the scale of HPC systems
requires the aggregation of larger number of individual components.  More
components means more frequent failures.  Current systems use powerful
error-correcting codes~(ECC), e.g., chipkill-correct, to protect against DRAM
errors.  However, chipkill-correct (and other similar techniques) require the
activation of a large number of memory devices (four times more than
less-protective techniques like single error correct double error
detect~(SECDED))~\cite{Jian13}.  Activating more memory devices requires more
power for each memory access.  However, because of tightening power budgets on
next-generation systems~\cite{Bergman08exascalecomputing} and technology
changes like High-Bandwidth Memory~(HBM)~\cite{HBM}, it is not yet clear that
chipkill-correct will continue to be viable.  Reduced device-feature sizes also
have the potential to result in more frequent failures.  Understanding the
implications of these trends requires that we have detailed knowledge of how
failures affect current leadership-class systems.

\sll{This is pretty detailed to include in the introduction.  Maybe a background
section?}
Error detection and handling is critical to pinpointing failing components and
taking corrective action in a timely fashion.  Error handling is typically
a cooperative activity between the platform hardware, firmware (UEFI or BIOS),
and the host operating system.  Errors are signaled in host firmware
or the OS directly. The firmware reads the hardware registers (\sll{of the CPU??}) 
and analyzes the component that generated the error in an effort to assess the severity of
the error.  Firmware then creates a detailed description of the error and
notifies the OS of its occurrence. Additionally, the host firmware (\sll{is ``firmware''
different than ``host firmware''}) may communicate
this error to the baseboard management controller~(BMC) for system management purposes.
When the error notification is signaled to the OS, either directly or from host
firmware, the OS typically inspect hardware registers or firmware and initiate
corrective action.  Time spent handling an error by the OS and/or firmware can perturb
application progress, even in cases where the failure does not initiate a
restart.

\sll{I \emph{think} that we're really only talking about memory errors?  Can we say that?}
Errors are typically classified into three categories: Correctable,
Uncorrectable, and Fatal. Correctable errors (CE) are errors that can be
corrected or mitigated in hardware such that the platform's state is the same as it
would have been left if no error had occurred.  An example of a CE is is a single bit 
(or single symbol in the case of chipkill) error.  Detected, uncorrectable errors (DUE) are those
errors that could be detected by hardware, but could not be corrected.
Multi-bit/Multi-symbol errors are an example of such an error.  In many cases
it is possible for the system to continue functioning in the face of these
errors, perhaps with a certain amount of lost state.  \sll{Not sure what the ``In many
cases'' caveat means.  If it can't continue functioning then isn't it a \emph{fatal}
error?}  Finally, fatal errors corrupt the state of the platform such that continued 
correct operation can no longer be guaranteed.  Recovering from a fatal error typically 
requires a full system halt and a reboot.

Recent research has largely focused on uncorrectable and fatal errors: errors
that require applications to be restarted.  However, the impacts of the most
common type of errors in large-scale systems, correctable errors, have largely
been overlooked. An analysis of failures on recent leadership-class systems
shows that the correctable failure rates are 20 times higher than uncorrectable
errors~\cite{meza:2015:revisiting}.  Correctable errors are typically handled at
a hardware level and are largely invisible to the application.  While the
application can continue to make progress despite the presence of correctable
errors (i.e. restarting the application is unnecessary), the time required to
correct and log these errors has the potential to impact application performance
by delaying application computation. These CEs can be transient, generating only
one or a small handful of error events on a node, or can be persistent and
generate in some cases millions of error events.  \sll{Struggling with the
significance of the difference between transient and persistent CEs.} These
\emph{bursty} CE events can lead to significant application slowdowns even on
current systems~\cite{BURSTY}.  \sll{\emph{bursty} still feels out of place to
me.  Fundamentally, we only care about rates, not about whether there are
``bursts'' of errors.  It's also not clear to me from this paragraph what we
mean by ``bursty''.}

\let\workinterval\relax
\let\ckpttime\relax
\let\txdelay\relax
\let\msgtime\relax
\let\minheight\relax
\newcommand{\workinterval}{1.25cm}
\newcommand{\ckpttime}{0.625cm}
\newcommand{\txdelay}{2.0mm}
\newcommand{\msgtime}{\workinterval+\txdelay}
\newcommand{\minheight}{0.5cm}
\usetikzlibrary{positioning}

\tikzstyle{position}=[fill=none,text=white,draw=none]
\tikzstyle{proc}=[fill=none,text=black,draw=none,shape=rectangle]
\tikzstyle{origtotal}=[fill=none,text=black,draw=black,shape=rectangle,
                       minimum width=3*\workinterval+2*\txdelay,
                       minimum height=\minheight]
\tikzstyle{coordtotal}=[fill=none,text=black,draw=black,shape=rectangle,
                        minimum width=3*\workinterval+\ckpttime+2*\txdelay,
                        minimum height=\minheight]
\tikzstyle{uncoordtotal}=[fill=none,text=black,draw=black,shape=rectangle,
                          minimum width=3*\workinterval+2*\ckpttime+2*\txdelay,
                          minimum height=\minheight]
\tikzstyle{ckpt}=[fill=black,text=white,draw=none,shape=rectangle,
                  minimum width=\ckpttime,minimum height=\minheight]
\tikzstyle{stall}=[fill=black!20,text=white,draw=black,shape=rectangle,
                   minimum height=\minheight]

\begin{figure*}[bt]
\centering{
\subfloat[without CE activity]{
\resizebox{0.25\textwidth}{!}{
\begin{tikzpicture}[semithick]

\node [proc] (p0) {$p_0$};
\node [position, right= 0.125cm of p0, minimum width=3*\workinterval+2*\ckpttime+2*\txdelay] (n0) {};
\node [origtotal, right= 0.125cm of p0]  (v1) {};
\node [position, right= \msgtime of v1.north west ]  (v2) {};
\node [position, right= \msgtime of v2.west ]  (v4) {};

\node [proc, below = 5mm of p0] (p1) {$p_1$};
\node [origtotal, right= 0.125cm of p1]  (v5) {};
\node [position, right= \msgtime of v5.west ]  (v6) {};
\node [position, right= \workinterval of v6.south west ]  (v8) {};

\node [proc, below of=p1] (p2) {$p_2$};
\node [origtotal, right= 0.125cm of p2]  (v9) {};
\node [position, right= \msgtime of v9.west ]  (v10) {};
\node [position, right= \msgtime of v10.west ]  (v11) {};

\draw [->, thick] (v1.south west) ++(\workinterval,0) -- ++(\txdelay, -5.00mm) 
      node[above right] {$\mathbf{m}_1$};
\draw [->, thick] (v8.south west) -- ++(\txdelay, -5.00mm) node[above right] {$\mathbf{m}_2$};

\draw [thick, dashed] (v2.north west) ++(0.0cm, 5.0mm) -- (v10.south west) -- ++(0, -5.0mm) node[below] {$t_1$};
\draw [thick, dashed] (v4.north west) ++(0.0cm, 5.0mm) -- (v11.south west) -- ++(0, -5.0mm) node[below] {$t_2$};
\end{tikzpicture} 
}\label{subfig:no_ckpt}
}
%
%
%\subfigure[uncoordinated checkpointing]{
\subfloat[with local CE activity delays]{
\resizebox{0.25\textwidth}{!}{
\begin{tikzpicture}[semithick]

\node [proc] (p0) {$p_0$};
\node [uncoordtotal, right= 0.125cm of p0]  (v1) {};
\node [position, right= \msgtime+\ckpttime of v1.north west ]  (v2) {};
\node [ckpt, right= \workinterval of v1.west ]  (v3) {$\delta$};
\node [position, right= \msgtime+\ckpttime of v2.west ]  (v4) {};

\node [proc, below of=p0] (p1) {$p_1$};
\node [uncoordtotal, right= 0.125cm of p1]  (v5) {};
\node [position, right= \msgtime+\ckpttime of v5.west ]  (v6) {};
\node [stall, minimum width=\ckpttime, left= 0.0cm of v6.west ]  (v7) {};
\node [ckpt, right= 0.00cm of v6.west ]  (v7) {$\delta$};
\node [position, right= \workinterval+\ckpttime of v6.south west ]  (v8) {};

\node [proc, below of=p1] (p2) {$p_2$};
\node [uncoordtotal, right= 0.125cm of p2]  (v9) {};
\node [position, right= \msgtime+\ckpttime of v9.west ]  (v10) {};
\node [position, right= \msgtime+\ckpttime of v10.west ]  (v11) {};
\node [stall, minimum width=2*\ckpttime, left= 0.00cm of v11.west ]  (v12) {};

\draw [->, thick] (v1.south west) ++(\workinterval+\ckpttime,0) -- ++(\txdelay, -5.00mm) 
      node[above right] {$\mathbf{m}_1$};
\draw [->, thick] (v8.south west) -- ++(2.0mm, -5.00mm) node[above right] {$\mathbf{m}_2$};

\draw [thick, dashed] (v2.north west) ++(0.0cm, 5.0mm) -- (v10.south west) -- ++(0, -5.0mm) node[below] {$t_1+\delta$};
\draw [thick, dashed] (v4.north west) ++(0.0cm, 5.0mm) -- (v11.south west) -- ++(0, -5.0mm) node[below] {$t_2+2\delta$};
\end{tikzpicture} 
}\label{subfig:uncoord_ckpt}
}
}
\caption{
        Example of how delays introduced by local correctable error (CE)
        activities may propagate along application communication dependencies.
        The processes $p_1$, $p_2$, and $p_3$ exchange two messages $m_1$ and
        $m_2$ in each of the three scenarios. The black regions marked with a
        white $\delta$ denote the execution of CE mitigation activities.  The
        grey regions denote periods in which the execution of a process is
        stalled due to an unsatisfied communication dependency.
}\label{fig:propagation}
\end{figure*}


In this paper, we present a detailed analysis of the relationship between the cost of
correcting and logging memory CEs and application performance on large-scale systems.
Specifically, to better understand the potential performance impact of CEs, we answer the 
following key questions:

\begin{itemize}
  \item What is the expected performance impact of CEs on applications running on current and 
        projected future extreme-scale systems?
  \item How frequently can CEs occur without significantly degrading application performance?
  \item What is the application performance impact of CEs that are isolated to a single process?
  \item How can system designers address CEs to improve application performance on next-generation systems?
\end{itemize}
\sll{For reference, these are the existing questions.}
\begin{itemize}
        \item At what CE frequency does one "Bursty" node have significant
              performance impact on current and future systems?
        \item Given current correctable error rates, what is the performance
              overheads of CE on current and expected future extreme-scale systems?
        \item How much can CE rates increase without significantly impacting
              performance?
        \item Given these performance overheads, what advice can we give system
              designers to keep slowdowns due to CEs low?
\end{itemize}

We anticipate that the relationship between application performance and CE 
overheads will become increasingly more important as error rates increase on
next-generation extreme-scale systems.  Dramatic growth in system size (both
in terms of node count and node density) and total memory volume are likely
to lead to more frequent CEs on future systems. 
\kbf{beef up justification for following sentence to justify increased rates}
Additionally, smaller feature sizes, manufacturing variability, changing memory
protection technology, hardware aging effects, and sub-threshold logic have the
potential to further increase the rate of CEs.  
More specifically, using a wide array of
applications, we demonstrate that the overheads associated with  correctable
errors can have a significant impact on overall application performance.  These
local correctable errors introduce overheads that can amplified or absorbed
globally by an application depending on an application's communication
activities. \sll{These last two sentences seem to be repeating content from 
elsewhere in the section.}

The introduction of delays in application processes by CE-related correction and 
logging activities is analogous to how operating system noise (or \emph{jitter}) can 
affect the performance of large-scale applications~\cite{Hoefler:2010:Characterizing, 
Ferreira:08:characterizing}.  \Cref{fig:propagation} illustrates this phenomenon.  
\Cref{subfig:no_ckpt} shows a simple application running across three processes 
($p_0$, $p_1$, and $p_2$) in the absence of CEs.  These three processes exchange two 
messages, $\mathbf{m}_1$ and $\mathbf{m}_2$.  For the purposes of this figure, we assume 
that these messages represent strict dependencies: any delay in the arrival of a message requires the
recipient to stall until the message is received. \Cref{subfig:uncoord_ckpt}
illustrates the potential impact of CE correction and logging.  If $p_0$
encounters a CE at the instant before it would have otherwise sent $\mathbf{m}_1$,
then $p_1$ is forced to wait (the waiting period is shown in grey) until the
message arrives.  If $p_1$ subsequently encounters a CE before sending
$\mathbf{m}_2$, then $p_2$ is forced to wait.  Part of the time that $p_2$
spends waiting is due to a delay that was originated by $p_0$, which it does not
communicate with.  The key point is that delays due to CEs can propagate based on 
communication dependencies in the application.

Based on the studies presented in this paper, we make the following
contributions \kbf{Contributions need revisiting}:
\sll{Do we need contributions?  It seems like we should include either the
questions above or the contributions below.}

\begin{itemize}

\item we demonstrate the impacts of the hardware/firmware/OS activities
        associated with correctable errors, and how those activities can changes
        based on platform configuration \S{?};

\item we analyze the potential impacts of this correctable errors on application
        performance.  We demonstrate that he performance impact of logging
                correctable DRAM errors is modest at error rates observed on
                current system.  In addition, we show these impacts change with
                increased error rates \S\S{?};

\item we show how an application's scale and communication pattern can dictate
        whether local overheads due to correctables are amplifies or absorbed by
        other processes \S\S{?}; and

\item we show the interplay between the correctable error rate and the duration
        of each mitigation activity, providing prescriptive advice on how best
        to reduce overheads due to these common errors on leadership-class platforms
        \S{?}.

\end{itemize}

Overall, this work provides critical analysis and insight into the overheads of
common correctable errors and provides practical advice to users and systems
administrators in an effort to fine-tune performance to application and system
characteristics.



\section{Experiential Approach}
\label{sec:approach}

\subsection{APEI Injection Framework}

\subsection{Simulating Correctable Overheads}

In general, the communication structure of Message Passing Interface (MPI)
programs cannot be determined offline because message matches cannot be
established statically~\cite{bronevetsky2009communication}.  This makes
modeling application performance analytically challenging even if all
parameters of the application (e.g., the complete communication structure and
all relative inter-process timings) are known.  We therefore use a validated
discrete-event simulation framework to evaluate the impact of local
correctable error mitigation activities on the performance of real applications.
%for real applications via their message traces.

Our simulation-based approach models correctable error mitigation activities as CPU detours: periods
of time during which the CPU is taken from the application and used to compute and commit
checkpoint data.  This approach allows a level of
fidelity and control not always possible in implementation-based approaches. It
also allows us to examine simulated systems much larger than those generally available.

Our simulation framework is based on the freely available
\LogGOPSim~\cite{Hoefler:2010:LogGOPSim} and the tool chain provided  by Levy et
al.~\cite{Levy2013UsingSimulation}.  \LogGOPSim uses the LogGOPS model, an
extension of the well-known LogP model~\cite{Culler:1993:LogP}, to account for
the temporal cost of communication events.  An application's communication
events are generated from traces of the application's execution.  These traces
contain the sequence of MPI operations invoked by each application process.
\LogGOPSim uses these traces to reproduce all communication dependencies,
including indirect dependencies between processes which do not communicate
directly.

\LogGOPSim can also extrapolate traces from small application runs; a trace
collected by running the application with $p$ processes can be extrapolated to
simulate performance of the application running with $k\cdot p$ processes. The
extrapolation produces exact communication patterns for MPI collective
operations and approximates point-to-point
communications~\cite{Hoefler:2010:LogGOPSim}.  The validation of \LogGOPSim and
its trace extrapolation features have been documented
previously~\cite{Hoefler:2010:LogGOPSim}, along with the simulators ability to
accurately predict application performance in the presence of performance
perturbations~\cite{Ferreira:2014:Understanding,Levy2013UsingSimulation,Hoefler:2010:Characterizing}

\subsection{Simulating Applications with Correctable Errors}

\kbf{Add application workloads}



\section{Results}
\label{sec:results}

In this section we outline the costs and impacts of DRAM correctable errors
using the test infrastructure structure outlined in the previous sections.  First,
we examine the correction, logging and decoding costs of these correctable
errors.  This characterization defines the cost of each DRAM error under a number
of different configuration.  Then, using this correctable error cost
characterization and the fault rate measured in a number of published studies as
a baseline, we demonstrate the performance impacts of correctable errors on
current and future systems.

\subsection{DRAM Correctable Costs}

In this section we outline the cost of DRAM correctable on a modern HPC system.
We are concerned with three costs: the hardware error correction costs (the cost
of the ECC code to correct), logging/decoding in the OS, and logging/decoding in
firmware.

To carry out this testing, we use the \blake system located at \detail{Sandia
National Laboratories}.  \blake is 48 node Intel Skylake system connected by
\kbf{ADD network detail}.  Each node consists of 4, 24 core, 2.1GHz Intel
Skylake processors (a total of 96 cores/node) and 175GB of DDR4 DRAM per node.
This system is running Red Hat Enterprise Linux Server release 7.4 and a Linux
3.10.693 version kernel.

To inject correctable DRAM errors we use the Error INJection (EINJ) facility of
the ACPI Platform Error Interface (APEI)~\cite{ACPISpec} described previously.
We also use the \selfish operating system noise (or \emph{jitter}) measurement
microbenchmark~\footnote{See appendix \Cref{sec:appendix} for further details on
these utilities}.

\begin{figure*}
\centering{
        \subfloat[Native OS Signature for \blake]{
                \includegraphics[ width=0.45\textwidth ]
                        {blake-native-selfish-cpu0_dtr}
                \label{blake:native}
        }
        \subfloat[``Dry Run'' Injection OS Signature ]{
                \includegraphics[ width=0.45\textwidth ]
                        {blake-10sec-dry_run-selfish-cpu0_dtr}
                \label{blake:dry-run}
        }
}
\caption{
        Native and ``dry run'' OS noise signature for \blake.  The ''dry run''
        option configures the EINJ interface at the requested frequency (in this
        case every ten seconds) but does not trigger the error.  This attempts
        to measure the cost of the error injection utility.  As can be seen from
        the figure, the injection utility impacts no additional OS noise
}
\label{fig:baselines}
\end{figure*}

As we are interested in measuring the noise signature of DRAM correctables, we
first must measure the native noise signature of the system, as well as ensuring
the error injection utility does not impart additional OS noise costs that would
not be part of an actual correctable DRAM error.  \Cref{fig:baselines} shows the
native OS noise signature on the \blake system and a ``dry-run'' execution of
our error injection utility.  The dry-run option of the injection utility
configures the injection interface to triggers errors at the requested interval,
in this case we arbitrarily chose every 10 seconds, but does not trigger the
error.  An OS noise event detected by \selfish is denoted by a spike in the
figure, the X-Axis being the time the noise event occurred and the duration of
the noise event being the amplitude on the Y-Axis.  From this figure we can see
that the native noise and dry-run signature are nearly identical.  Therefore the
injection framework does not impart any additional OS noise over the systems
native signature.  

\begin{figure*}
\centering{
        \subfloat[EDAC]{
                \includegraphics[ width=0.45\textwidth ]
                        {blake-10sec-edac-selfish-cpu0_dtr}
                \label{FF:EDAC}
        }
        \subfloat[EXTLOG]{
                \includegraphics[ width=0.45\textwidth ]
                        {blake-10sec-extlog-selfish-cpu0_dtr}
                \label{FF:EXTLOG}
        }
}
\caption{
Firmware first APEI ``DRAM correctable'' error injection at a 10 second interval
}
\label{fig:FF:injection}
\end{figure*}

\begin{figure*}
\centering{
        \subfloat[EDAC]{
                \includegraphics[ width=0.45\textwidth ]
                        {blake-10sec-edac-ignore_ce-selfish-cpu0_dtr}
                \label{ignore_ce:EDAC}
        }
        \subfloat[EXTLOG]{
                \includegraphics[ width=0.45\textwidth ]
                        {blake-10sec-extlog-ignore_ce-selfish-cpu0_dtr}
                \label{ignore_ce:EXTLOG}
        }
}
\caption{
Firmware first APEI ``DRAM correctable'' error injection at a 10 second interval with
        {\texttt{mca=ignore\_ce}, handling every $10^{th}$ error}
}
\label{fig:FF:ignore_ce}
\end{figure*}


%\input{lessons}


\section{Related Work}
\label{sec:related}

Efforts to characterize the frequency and type of correctable and uncorrectable
failures on large-scale HPC and cloud systems have been ongoing for over a
decade now.  Schroeder and Gibson studied failures in supercomputer systems at
LANL~\cite{Schroeder:2006:Large-scale}.  Schroeder {\it et al.}~conducted a
large-scale field study using Google's server fleet~\cite{Schroeder09}.  Li {\it
et al.}~studied memory errors on three different data sets, including a server
farm of an Internet service provider~\cite{Li07}.  In 2010, Li {\it et
al.}~published an expanded study of memory errors at an Internet server farm and
other data center sources~\cite{Li10}.  Hwang {\it et al.}~published an expanded
study on Google's server fleet, as well as two IBM Blue Gene
clusters~\cite{Hwang12}.  Sridharan and Liberty presented a study of DRAM
failures in a high-performance computing system~\cite{Sridharan12}.  El-Sayed
{\it et al.}~studied temperature effects of DRAM in data center
environments~\cite{Elsayed12}.  Similarly, Siddiqua {\it et al.}~studied DRAM
failures from client and server systems~\cite{Siddiqua13}.  Sridharan {\it et
al.}~studied DRAM and SRAM faults, with a focus on positional and vendor
effects~\cite{Sridharan13}.  Di Martino {\it et al.}~studied failures in Blue
Waters, an HPC system at the University of Illinois,
Urbana-Champaign~\cite{bluewaters}.  Finally, Bautista-Gomez {\it et
al.}~\cite{Bautista-Gomez:2016:Unprotected} presented a study of DRAM memory
errors on a large-scale system in the explicit absence of an ECC in an effort
understand the behavior of raw memory failures.

\kbf{Add letters paper and other SMM mode papers}

Our study has origins in previous works that characterizes application behavior
in the presence of OS
noise~\cite{Ferreira:2008:Characterizing,Hoefler:2010:Characterizing}.
Collectively, this research shows that the pattern of OS noise events determines
the impact on application performance and the benefits of coordination.
Moreover, it shows that the duration of an OS noise event can significantly
slowdown application performance.

\kbf{Add paragraph outlining distinction of this work}


%\input{conclusion}


% For peer review papers, you can put extra information on the cover
% page as needed:
% \ifCLASSOPTIONpeerreview
% \begin{center} \bfseries EDICS Category: 3-BBND \end{center}
% \fi
%
% For peerreview papers, this IEEEtran command inserts a page break and
% creates the second title. It will be ignored for other modes.
%\IEEEpeerreviewmaketitle

% use section* for acknowledgment
\opt{NotBlind}{
\section*{Acknowledgment}
\kbf{Add blurbs}
}

\bibliographystyle{IEEEtran}
\bibliography{all} 

\clearpage
\appendix
\section{Artifact Description: \myTitle}

\kbf{This is close, some details need to be verified}

%%%%%%%%%%%%%%%%%%%%%%%%%%%%%%%%%%%%%%%%%%%%%%%%%%%%%%%%%%%%%%%%%%%%%
\subsection{Abstract}

This artifact contains instructions on accessing the the following
data/utilities from the paper:
\begin{itemize}
        \item A script to properly configure the APEI error injection interface
                used in this work (if supported by platform)
        \item The error injection utility that injects errors at the period
                specified.
        \item The \selfish utility used to measure the overheads of the injected
                correctable errors.
        \item The \LogGOPSim simulator and application execution traces used for
                the correctable DRAM error logging impacts  experiments 
        \item The appropriate analysis and plotting scripts used in the paper
\end{itemize}

%%%%%%%%%%%%%%%%%%%%%%%%%%%%%%%%%%%%%%%%%%%%%%%%%%%%%%%%%%%%%%%%%%%%%
\subsection{Description}

\subsubsection{Check-list (artifact meta information)}

{\small
\begin{itemize}
        \item {\bf Program}: C++ for \LogGOPSim, C and {\texttt{cmake}} for
                \selfish, Python (numpy and matplotlib) for error injection, and
                {\texttt{bash}} for APEI configuration script.
        \item {\bf Compilation}: g++ and gcc 4.2+ for \LogGOPSim and \selfish,
                Python 2.7+ for the injection and analysis scripts.
        \item {\bf Data set}: In this paper, we analyzed a collection of application
                execution traces in our correctable DRAM error experiments.  The
                applications used were: LAMMPS-lj, LAMMPS-crack, LAMMPS-SNAP,
                LULESH, miniFE, SPARC, and CTH.  In addition, we also used execution
                traces for three microbenchmarks: allreduce, reduce, and stencil
                \kbf{Ensure these are correct}.
        \item {\bf Run-time environment}: Any modern Linux distribution
        \item {\bf Hardware}: Any for \LogGOPSim.  For DRAM Error INJection
                (EINJ) testing, a platform that support ACPI 5.0+ is needed.
        \item {\bf Experiment customization}: modify Makefile to specify different
                application traces.
        \item {\bf Publicly available?}: All but two of the execution traces
                (CTH and SPARC) is publicly  available.  Because CTH and SPARC
                are export-controlled, their traces cannot be released publicly.  
\end{itemize}
}

\subsubsection{How software can be obtained (if available)}
\begin{itemize}
\item \LogGOPSim is available at:\\
    \url{https://htor.inf.ethz.ch/research/LogGOPSim/LogGOPSim-1.1.tgz}.  

\item Error injection and \selfish scripts can be found in our repository at: 
    \detail{made up repo}
\end{itemize}

\subsubsection{Hardware dependencies}

\LogGOPSim will run on any modern general-purpose computer. However, the
largest application scale and duration that can be simulated depends on
available RAM.  Memory usage increases linearly in both and scale and application
runtime.

The hardware error injection scripts can only be run on a platform which fully
support the APEI interface defined in ACPI 5.0+ specification.  The majority of
modern production hardware the authors has access to \emph{do not} fully support
this specification.  Included in the repo (and detailed below) is a script to
test and configure a platform for this testing, if supported.

\subsubsection{Software dependencies}

Application simulations were run on a supercomputer running Linux located at
\detail{Sandia National Laboratories}. \LogGOPSim and the analysis scripts have
been tested on Linux and MacOS running Python 2.7, Matplotlib 2.0.0, and g++/gcc
4.2.1  The error injection scripts must be run on a modern Linux with a kernel
version of 3.10 or greater.

\subsubsection{Datasets}

\begin{table}
\centering
\begin{tabular}{ l c }
\toprule
LogGOPS parameter & Value\\
\midrule
\textcolor{red}{L}atency                & 1.8$\mu s$ \\
\textcolor{red}{o}verhead per message   & 12.4$\mu s$ \\
\textcolor{red}{g}ap per message        & 2.6$\mu s$  \\
\textcolor{red}{G}ap per byte           & 1$ns$     \\
\textcolor{red}{O}verhead per byte      & 0$ns$     \\
\textcolor{red}{S}: rendezvous threshold  & 65,536 bytes \\
\bottomrule
\end{tabular}
\caption{
  LogGOPS parameters used in our study. Values measured using~\cite{netgauge-web}
}
\label{tab:logp}
\end{table}

Application traces used (excluding CTH and SPARC) are located in the {\texttt{traces}} 
directory of our repository.  LogP parameters used are specified in \reftab{tab:logp}.

%%%%%%%%%%%%%%%%%%%%%%%%%%%%%%%%%%%%%%%%%%%%%%%%%%%%%%%%%%%%%%%%%%%%%
\subsection{Installation}

Installation instructions for \LogGOPSim is available at: \url{https://htor.inf.ethz.ch/research/LogGOPSim/}

Error injection and analysis utilities installation instruction:


\begin{Verbatim}[commandchars=\\\{\},frame=single]
 $ git clone \detail{A very fake repo}
\end{Verbatim}

\subsubsection{APEI Error INJection Installation (EINJ)}

The {\texttt{configure\_apei.sh}} script checks and properly installs system
software for error injection.  If script fails on any test, injection support is
not supported on architecture.  Script should be run on a Linux system, and with
{\texttt{root}} privileges.

\begin{Verbatim}[commandchars=\\\{\},frame=single]
 $ sudo configure_apei.sh
Checking kernel version: 3.10.0-229 \color{green}{PASS}
Checking kernel commandline:        \color{green}{PASS}
Checking for ACPI EINJ table: found \color{green}{PASS}
Checking for kernel modules:
 ... loading einj module            \color{green}{PASS}
Checking supported APEI error injection
types:
 0x00000008 Memory Correctable
 0x00000010 Memory Uncorrectable
non-fatal
 0x00000020 Memory Uncorrectable fatal
 0x00000040 PCI Express Correctable
 0x00000080 PCI Express Uncorrectable
non-fatal
 0x00000100 PCI Express Uncorrectable
fatal
                                    \color{green}{PASS}
Test Complete: \color{green}{PASS}
\end{Verbatim}


\subsubsection{Error Injection Utility}

If the {\texttt{configure\_apei.sh}} script from the previous sections completes
without error, error injection is possible on the platform.  The
{\texttt{error\_inject.py}} script included in the repo can be used to to inject
correctable (and detected, non -fatal uncorrectables as well).  Passing {\texttt{--help}}
to the script provides a descrition of the options (must be run as
{\texttt{root}} as well)

\begin{Verbatim}[commandchars=\\\{\},frame=single]
 $ ./error_inject.py --help
usage: error_inject.py [-h] [-i INTERVAL]
[-l LENGTH] [-c | -u] [-d] [-v]
[-a ADDR [ADDR ...]]

APEI error injection utility

optional arguments:
  -h, --help show this help
message and exit
  -i INTERVAL, --interval INTERVAL
Periodic interval in seconds to inject
memory errors
  -l LENGTH, --length LENGTH Length
in seconds to run application (default
1200 secs)
  -c, --dram-correctable Inject
DRAM correctable error (default on)
  -u, --dram-uncorrectable Inject
DRAM uncorrectable error non-fatal
(default off)
  -d, --dry-run         Dry-run,
do not actually inject the error
(default off)
  -v, --version         show
program's version number and exit
  -a ADDR [ADDR ...],
  --addr ADDR [ADDR ...] List
of address to inject into
(default 0x12345000)
 
\end{Verbatim}

As an example, if we want to inject correctable memory errors at an interval of
every 100 seconds, at addresses {\texttt{0x12345000}}, {\texttt{0x12346000}} and
{\texttt{0x123457000}} for 1400 seconds.

\begin{Verbatim}[commandchars=\\\{\},frame=single]
 $ sudo ./error_inject.py -i 100 -l 1400
-c -a 0x12345000 0x12346000 0x123457000
\end{Verbatim}

\subsubsection{Correctable Impacts Measurement}

The \selfish utility included in the repo can be used to measure the impact of
correctable errors.  \selfish is a C program that depends on the
{\texttt{gengetopt}}
utility~\footnote{\url{https://www.gnu.org/software/gengetopt/gengetopt.html}},
a C99 compliant compiler, {\texttt{cmake}}, and an MPI library. \selfish is
located in the {\texttt{tools/selfish}} directory of the repo.  Building and
running is done as follows:


\begin{Verbatim}[commandchars=\\\{\},frame=single]
 $ cd tools/selfish
 $ mkdir build
 $ cd build
 $ cmake ..
 -- Found gengetopt: /opt/local/bin/
gengetopt (2.22.6)
 -- Performing Test HAVE_NB_ALLREDUCE -
Success
 -- Have MPI_Iallreduce(), we are good
 -- Configuring done
 -- Generating done
 -- Build files have been written to:
${PWD}/build
 $ make 
Scanning dependencies of target selfish
 [ 83%] Building C object src/CMakeFiles/
selfish.dir/cmdline.c.o
 [ 91%] Building C object src/CMakeFiles/
selfish.dir/selfish.c.o
 [100%] Linking C executable selfish
 [100%] Built target selfish

\end{Verbatim}

Passing the {\texttt{--help}} option to \selfish gives command line options

\begin{Verbatim}[commandchars=\\\{\},frame=single]
 $ selfish --help
 usage: selfish [-h] [-t THRESHOLD]
[-l LENGTH]

OS jitter recording microbenchmark

  -h, --help Print help and exit
  -t, --threshold=INT Threshold.
Values less then are not recorded
(default= 150 nanoseconds)
  -l, --length=INT Collection duration
in seconds (default= 1200 seconds)

\end{Verbatim}

As an example, if we want to collect \selfish traces in $24$ cores on a node for
1200 seconds, and using the default $150$ nanosecond threshold.

\begin{Verbatim}[commandchars=\\\{\},frame=single]
 $ mpirun -n 24 ./selfish -l 1200

 Calibrated tps: [ 2.1e+09, 2.1e+09,
2.1e+09 ]

 Avg Bench Wall Time: 1200.44 secs.
 Avg File Write Time: 0.36 secs.
 Avg Total Kernel Time: 1200.00 secs.

 Workload: Selfish.
 Problem description: rtsc() threshold
timing
 Threshold: 150 nsecs (cycles: 314)

 Datapoints: [ 926097, 966200, 996097 ]
\end{Verbatim}

%%%%%%%%%%%%%%%%%%%%%%%%%%%%%%%%%%%%%%%%%%%%%%%%%%%%%%%%%%%%%%%%%%%%%
\subsection{Experiment workflow}

\begin{Verbatim}[commandchars=\\\{\},frame=single]
  $ make DRAM-overheads # For \LogGOPSim
runs and correctable DRAM plots
\end{Verbatim}

\begin{Verbatim}[commandchars=\\\{\},frame=single]
  $ make analysis # For analysis plots
\end{Verbatim}

%%%%%%%%%%%%%%%%%%%%%%%%%%%%%%%%%%%%%%%%%%%%%%%%%%%%%%%%%%%%%%%%%%%%%
\subsection{Evaluation and expected result}

All generated figures (excluding those from export-controlled applications) will
be placed in the {\texttt{figs}} directory

%%%%%%%%%%%%%%%%%%%%%%%%%%%%%%%%%%%%%%%%%%%%%%%%%%%%%%%%%%%%%%%%%%%%%
\subsection{Experiment customization}

New application traces can be added for analysis by adding the new trace
directory to the {\texttt{APPLICATION\_TRACES}} variable in the top-level Makefile

New DRAM correctable noise traces can be added by modifying the
{\texttt{NOISE\_TRACES}} variable in that same Makefile

%%%%%%%%%%%%%%%%%%%%%%%%%%%%%%%%%%%%%%%%%%%%%%%%%%%%%%%%%%%%%%%%%%%%%
\subsection{Notes}

Using \LogGOPSim to simulate the execution of an application running on 
a large number of nodes can be very time-consuming.  In extreme cases,
it may take a day or more to simulate a few minutes of execution of
very large applications.

\clearpage
\section{Artifact Evaluation: \myTitle}

\subsection{Formula's Used for \Cref{tab:CE_rate}}


\end{document}
